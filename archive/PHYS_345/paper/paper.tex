\documentclass[12pt]{spieman}  % 12pt font required by SPIE;
\usepackage{amsmath,amsfonts,amssymb}
\usepackage{graphicx}
\usepackage{tocloft}

\title{Determining the value of Planck's Constant using LEDs}

\author{Johannes Byle}
\affil{Wheaton College, Physics and Astronomy, 501 College Ave, Wheaton, USA, 60187}

\renewcommand{\cftdotsep}{\cftnodots}
\cftpagenumbersoff{figure}
\cftpagenumbersoff{table}
\begin{document}
    \maketitle

    \begin{abstract}
        % Discussion of Method
        The fundamental physical principles by which LEDs operate are completely quantum mechanical, making it
        possible to observe quantum effects using macroscopic measurements.
        Using a dataset of the knee voltages of LEDs at different wavelengths, the value of Planck's constant is
        extracted by curve fitting and compared to the expected value.
        % Discussion of Results
        The results of this curve fitting show that LEDs can be used to determine the value of $h$, as the
        relationship between the knee voltage and wavelength matches the quantum mechanical model.
        However, the value of $h$ extracted from these measurements differs significantly from expected value.
        % Conclusion
        Therefore, though it is seemingly possible to extract Planck's constant from the knee voltage of an LED, this
        data set probably has some significant systematic error.
    \end{abstract}

    \keywords{LED, Planck's Constant}

    {\noindent \footnotesize\textbf{*}Johannes Byle, \linkable{johannes.byle@my.wheaton.edu}}

    \begin{spacing}{2}

        \section{Introduction}\label{sec:intro}
        % Brief review of topic
        The ubiquity of devices that rely on quantum effects in our daily lives makes it relatively easy to measure
        constants without the need for specialized equipment.
        One such measurement is the measurement of Planck's constant using a light emitting diode or LED\@.
        Because the wavelength of an LED is determined solely by the band gap of the material, the minimum voltage
        needed for current pass through it (the knee voltage) is directly related to the wavelength of the LED\@.
        The following equation governing this relationship was discovered by Russian scientist Oleg Vladimirovich
        Losev long before the physics of LEDs was fully understood.\cite{Zheludev_2007}
        \begin{equation}
            \label{eq:master}
            v=\frac{eV}{h}
        \end{equation}
        % Motivation of the work
        This equation makes it possible to measure the constants of quantum mechanics outside of the physics laboratory.
        % Summary of the model and method used in your work
        By measuring the knee voltage of LEDs of different wavelengths it is possible to extract the value for $h$ by
        curve fitting.
        % Summary of the model and method used in your work
        Ultimately, this makes it possible to compare whether or not this experimental value of $h$ is in agreement
        with the known value of $h$.


        \section{Theoretical Background}\label{sec:theoretical-background}
        Ideal LEDs are simply diodes;
        when a current is applied in one direction, provided it has sufficient voltage, it will cause electrons to
        jump across the band gap and when these electrons recombine with the holes they left behind they release a
        photon.\cite{GAYRAL2017453}
        The energy of this photon is thus determined by the band gap, and therefore each LED will emit light at a
        single wavelength determined by its band gap.
        However, because LED are made of real materials there are several sources of uncertainty.
        For instance, because these LEDs are not operated at 0 K, and because the band gap of real materials isn't a
        simple constant, there will be some broadening of the band gap, and thus the wavelength will not be a single
        value but rather range of values.\cite{brown2019iii}
        However, equation~\eqref{eq:master} is still valid as this broadening is small relative to the other sources
        of error.
        The only changes that need to be made are accounting for $V_0$ and converting from frequency to wavelength.
        \begin{equation}
            \label{eq:middle}
            \frac{c}{\lambda}=\frac{e(V+V_0)}{h}
        \end{equation}
        Rearranging to make $V$ a function of $\lambda$:
        \begin{equation}
            \label{eq:final}
            V=\frac{hc}{e\lambda}-V_0
        \end{equation}
        The values for $c$, the speed of light and $e$, the electron charge, are 299792458 m and $1
        .602176634\times10^{-19}$ C respectively.\cite{Newell_2018}
        Making $V_0$ and $h$ fitting parameters will allow us to extract these values from the data set.
        The value of $h$ extracted from the curve fit will be compared to the known value of $6
        .62607004\times10^{-34}$ m$^2$ kg / s.\cite{Newell_2018}


        \section{Methods}\label{sec:methods}
        The LED data was collected in two different sets using two different methods.
        In the first set the knee voltage was measured by measuring when current began flowing through the LED using an
        oscilloscope.
        In the second set the knee voltage was measured by measuring the intensity of the light emitted by the LED\@.
        In theory, both of these methods should produce identical results, however it is possible that current and
        intensity are not linearly related, and thus the broadening in the band gap may produce systematic
        differences between the two measurements.

        Both data sets were analyzed identically;
        the voltage and wavelength data was fit to equation~\eqref{eq:final} with the inverse of the voltage
        uncertainties used as weights for the fitting function.
        The values for the uncertainties of the fitting parameters were calculated from the square root of the
        diagonal of the covariance matrix.
        The probability of discrepancy was calculated using the following equations:
        \begin{equation}
            \label{eq:t-value}
            t=\frac{h_{\text{fit}}-h}{\sqrt{\delta h_{\text{fit}}^2-\delta h^2}}
        \end{equation}
        \begin{equation}
            \label{eq:prob}
            \text{Probability of discrepancy}=\int_{-t}^{-t}\frac{1}{2\pi}{-\frac{1}{2}x^2}dx
        \end{equation}


        \section{Results}\label{sec:results}

        The curve fits, as plotted below, demonstrated that the relationship between LED wavelength and knee voltage
        was governed by equation~\eqref{eq:master}.
        However, the extracted values for $h$ were different from the expected value, and were also significantly
        different from each other, as can be seen in table~\ref{tab:results}.
        This suggests that there is some systematic error in the data;
        either from the broadening of the band gap, or from systematic errors in the experimental setup.
        These errors could also account for the large difference between the measurements.

        \begin{figure}[H]
            \label{fig:figure1}
            \begin{center}
                \begin{tabular}{c}
                    \includegraphics[height=5.5cm]{set.eps}
                \end{tabular}
            \end{center}
            \caption {Plot of the data sets and fits. The lines are the plots of equation~\eqref{eq:final} with the
            fitting parameters from the respective dataset.}
        \end{figure}

        \begin{table}[H]
            \caption{The fitting parameters extracted from the data. $h$ is the value of Planck's constant extracted
            through the fit, $\delta h$ is the uncertainty in $h$, and the probability column is the probability that
            the expected and observed values of $h$ are in fact different.}
            \label{tab:results}
            \begin{center}
                \begin{tabular}{|l|l|l|}
                    \hline
                    & $h$                                     & probability of discrepancy \\
                    \hline
                    set 1 & $7.64\pm 0.5\times 10^{-34}$ J$\cdot$ s & 0.37                       \\
                    set 2 & $9.91\pm 0.5\times 10^{-34}$ J$\cdot$ s & 0.40                       \\
                    \hline
                \end{tabular}
            \end{center}
        \end{table}


        \section{Conclusion}\label{sec:conclusion}
        This work demonstrated that the value of Planck's constant can be extracted from the knee voltage of LEDs.
        However, the difference between the two datasets, as well as the large uncertainty of the data, suggests that
        there is some systematic error in the measurements.
        Thus, although it seems LED knee voltages are governed by equation~\eqref{eq:master} this data does not
        conclusively confirm that there is a difference in the value of $h$.
        Future work would benefit from examining the voltage dependence of the luminescence of the LEDs for a range
        on the electromagnetic spectrum in order to more accurately determine the wavelength of the LED\@.

    \end{spacing}


    \bibliography{report}   % bibliography data in report.bib
    \bibliographystyle{spiejour}   % makes bibtex use spiejour.bst


\end{document}