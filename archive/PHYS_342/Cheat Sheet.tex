\documentclass[1pt]{report}
\usepackage{multicol}
\setlength{\columnseprule}{0.2pt}
\usepackage[T1]{fontenc}
\usepackage[document]{ragged2e}
\usepackage[latin9]{inputenc}
\usepackage[left=.5cm, right=.5cm, top=.5cm]{geometry}
\usepackage{babel}
\usepackage{enumitem}
\usepackage{graphicx}
\begin{document}
\begin{multicols}{3}
\textbf{Fundamental Inputs}
\begin{itemize}[noitemsep]
\item Principle of Superposition
\item Coulombs Law ($F=\frac{1}{4\pi\epsilon_0}$)
\end{itemize}
\begin{flushleft}
$\oint_S {E_n dA = \frac{1}{{\varepsilon _0 }}} Q_{inside}$
\linebreak
$\nabla\cdot E=\frac{1}{\epsilon_0}\rho$|$V(r)=-\int^rE\cdot dl$
\linebreak
$E=-\nabla V$|$C=\frac{Q}{V}$
\linebreak
$W=\frac{1}{2}CV^2$|$W=\frac{\epsilon_0}{2}\int E^2d\tau$
\linebreak
$V(r)=\frac{1}{4\pi\epsilon_0}\int\frac{\rho(r)}{r_s}d\tau$
\linebreak
Potential is continuous at boundaries 
\textbf{Method of Images}
\begin{itemize}[noitemsep]
\item Replace the conducting plane with a mirror image charge
\item Use Gausss law on each charge in isolation
\item Sum up the electric field contribution from each charge
\end{itemize}
Poisson's equation:
$\nabla^2V=-\frac{1}{\epsilon_0}\rho$
\linebreak
Laplace's equation:
$\nabla^2V=0$
\linebreak
Laplace's equation in Cartesian:
$\frac{\delta^2V}{\delta x^2}\frac{\delta^2V}{\delta y^2}\frac{\delta^2V}{\delta z^2}=0$
\linebreak
Converted to PDE
\linebreak
$V(x,y)=(Ae^{kx}+B^{-kx})(Csin(ky)+Dcos(ky))$
Laplace's equation is true if $\rho$ is zero
\linebreak
\textbf{First Uniqueness Theorem:}
The solution to Laplace equation in some volume \textit{V} is uniquely determined if V is specified on the boundary surface \textit{S}
\textbf{Corollary:}
The potential in a volume \textit{V} is uniquely determined if (a) the charge density throughout the region, and (b) the value of V on all boundaries, are specified. 
\textbf{Second Uniqueness Theorem:}
In a volume \textit{V} surrounded by a conductors and containing a specified charge density \textit{$\rho$}, the electric field is uniquely determined if the \textit{total charge} on each conductor is given. (The region as a whole can be bounded by another conductor or else unbounded)
\textbf{Fourier s Trick}
$V_0(y) = \sum_{n=1}^{\infty} C_n \sin \left(\frac{n \pi y}{a} \right)$
\linebreak
$V_0(y) \sin \left(\frac{n' \pi y}{a} \right) =\sum_{n=1}^{\infty} C_n \sin \left(\frac{n \pi y}{a} \right)\sin \left(\frac{n' \pi y}{a} \right)$
\linebreak
$ \int_0^a V_0(y) \sin \left(\frac{n' \pi y}{a} \right) \, dy = \sum_{n=1}^{\infty} C_n \int_0^a  \sin \left(\frac{n \pi y}{a} \right)\sin \left(\frac{n' \pi y}{a} \right) \, dy = \frac{a}{2} C_{n'}$
\linebreak
$C_{n'} = \frac{2}{a} \int_0^a V_0(y) \sin \left(\frac{n' \pi y}{a} \right) \, dy$
\linebreak
\textbf{Legendre Polynomials}
\linebreak
$P_0(x) = 1$
\linebreak
$P_1(x) = x$
\linebreak
$P_2(x) = \frac{1}{2}(3x^2 - 1)$
\linebreak
$P_3(x) = \frac{1}{2}(5x^3-3x)$
\linebreak
$V(r,\theta)=\sum_{l=0}^{\infty}(A_lr^l+\frac{B^l}{R^{l+1}})P_l(cos\theta)$
\textbf{Monopole Expansions}
\linebreak
Monopole $(V~1/r)$
\linebreak
Dipole $(V~1/r^2)$
\linebreak
Quadrupole $(V~1/r^3)$
\linebreak
Octopole $(V~1/r^4)$
\linebreak
$\rho=\sum_{i=1}^{n}q_ir_i$
\linebreak
$V_{mon}(r)=\frac{1}{4\pi\epsilon_0}\frac{Q}{r}$
\linebreak
$V_{dip}(r)=\frac{1}{4\pi\epsilon_0}\frac{\rho\cdot\hat{r}}{r^2}$
\linebreak
$E_{dip}(r,\theta)=\frac{\rho}{4\pi\epsilon_0r^3}(2cos\theta\hat{r}+sin\theta\hat{\theta})$
\linebreak
Charge is evenly distributed across capacitor plates
\linebreak
$F=QE$
\linebreak
Volume Charge
$E(r)=\frac{1}{4\pi\epsilon_0}\int\frac{\rho(r)}{r^2}\hat{r}d\tau$
\includegraphics[scale=.5]{Griffiths.png}
\end{flushleft}
\end{multicols}
\end{document}
