\documentclass[12pt]{article}
\usepackage{graphicx}
\usepackage{xcolor}
\usepackage{listings}
\usepackage{amsmath}
\usepackage{subfiles}
\usepackage{circuitikz}
\author{Johannes Byle}
\begin{document}
    \title{HW Set 3}
    \maketitle

    \noindent
    \title{HW Set 3}
    \maketitle

    \noindent
    \textbf{1}\\
    \begin{circuitikz}[american,scale=0.75]
        \draw (0,0) to[sV=$V_{\omega=0}$] (0,3) to[R=$R$] (0, 6);
        \draw (0,6) to[short,-*] (6,6);
        \draw (6,6) to[short,-*] (9,6);
        \draw (6,6) to[R=$R$,-*] (6,0);
        \draw (0,0) to[short,-*] (9,0);
    \end{circuitikz}
    \[\frac{V_{out}}{V_{in}}=\frac{1}{2}\approx-3.01\text{ dB}\]
    \begin{circuitikz}[american,scale=0.75]
        \draw (0,0) to[sV=$V_{\omega=\infty}$] (0,3) to[R=$R$] (0, 6);
        \draw (0,6) to[short,-*] (6,6);
        \draw (6,6) to[short,-*] (9,6);
        \draw (6,6) to[R=$100R$] (6,3);
        \draw (6,3) to[R=$R$,-*] (6,0);
        \draw (0,0) to[short,-*] (9,0);
    \end{circuitikz}
    \[\frac{V_{out}}{V_{in}}=\frac{101}{102}\approx-0.04\text{ dB}\]
    \textbf{2} Its a low pass filter.
    \begin{gather*}
        H=\frac{\frac{1}{j\omega C}}{j\omega L+\frac{1}{j\omega C}}\\
        |H|=\frac{1}{\left(CL\omega^2-1\right)^2}\\
        \omega=\sqrt{1+\sqrt{2}}\sqrt{\frac{1}{CL}}=\sqrt{\frac{1-\sqrt{2}}{1\cdot1\cdot10^{-6}}}\approx1554\\
        f=247\text{ Hz}
    \end{gather*}
%    This is a high pass filter because it decreases the signal more significantly at low frequencies.
%    \[H=\frac{\left(\frac{1}{100R}+\frac{1}{2j\omega L}\right)^{-1}+R}{\left(j\omega C+\frac{1}{j\omega
%    L}\right)^{-1}+\left(\frac{1}{100R}+\frac{1}{2j\omega L}\right)^{-1}+2R}\]
    \textbf{3}\\
    \includegraphics[width=0.65\linewidth]{hw3_current.eps}\\
    \includegraphics[width=0.65\linewidth]{hw3_voltage.eps}
    \begin{lstlisting}[language=Python,label={lst:lstlisting}]
import numpy as np
import matplotlib.pyplot as plt
from scipy.integrate import quad

t = np.linspace(-2, 10, 10000)
v = np.zeros(np.shape(t))
v[(t < 1) & (0 < t)] = 40
r = np.zeros(np.shape(t))
r[(t <= 1) & (0 < t)] = 100
r[t > 1] = 200
L = 200


def integral(time):
    def f(x):
        if x < 0:
            V = 0
            R = 0
        elif 0 <= x < 1:
            V = 40
            R = 100
        else:
            V = 0
            R = 200
        return (np.exp((R * x) / L) * V) / L

    y, err = quad(f, 1, time)
    return y


f = np.array([integral(n) for n in t])
i = np.exp(-(r * t) / L) * (f - f[0])
plt.plot(t, i)
plt.plot(t, i * 50)
plt.show()

    \end{lstlisting}

    This code simply plots the following equation for the given parameters.
    \[e^{-\frac{Rt}{L}}\left(\int_{1}^{t}\frac{e^{\frac{R\xi}{L}}V\left(\xi\right)}{L}d\xi+C\right)\]
    The code could have been a lot shorter and simpler, but I had to write my own function for the integral portion of
    the equation.

    \textbf{4.a}
    \begin{gather*}
        \omega_r=\frac{1}{\sqrt{LC}}\\
        C=\frac{1}{L\omega_r^2}=\frac{1}{0.1\cdot100^2}=0.001\text{ F}\\
    \end{gather*}
    At this condition the imaginary component is zero, and its called the resonance frequency.\\
    \textbf{4.b}
    \begin{gather*}
        Q=\frac{\omega_{0}L}{R}\\
        R=\frac{\omega_0 L}{Q}
    \end{gather*}
    \includegraphics[width=0.65\linewidth]{hw3_e.eps}
    \begin{lstlisting}[language=Python,label={lst:lstlisting2}]
import numpy as np
import matplotlib.pyplot as plt

C = 0.001
L = 0.1
R_1 = (100 * L) / 0.1
R_2 = (100 * L) / 1
R_3 = (100 * L) / 10


def z(w, R):
    val = R + 1 / (1j * w * C) + 1j * w * L
    return (10 / np.sqrt(np.real(val * np.conj(val)))) / (np.sqrt(2))


w = np.linspace(10 ** (-2), 10 ** 6, 1000000)
plt.plot(w, z(w, R_1), label="Q=0.1")
plt.plot(w, z(w, R_2), label="Q=1")
plt.plot(w, z(w, R_3), label="Q=10")
plt.xlabel("Frequency (Hz)")
plt.ylabel("Current (A)")
plt.xscale("log")
# plt.yscale("log")
plt.legend()
plt.show()
    \end{lstlisting}
\end{document}