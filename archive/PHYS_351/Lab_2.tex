\documentclass[12pt]{article}

\usepackage{amsmath}

\author{Johannes Byle}
\begin{document}
%    Learning Outcome 1: What specifically did you learn about circuits or circuit
%   components (resistors, capacitors, inductors, diodes, transistors, op amps) that you did
%   not know coming into lab? Identify at least two examples.
%   2. Learning Outcome 2: What did you learn about one or more of the tools and
%   techniques listed in Learning Outcome 2 that you did not know coming into lab?
%   Identify at least two examples.
%   3. Learning Outcome 3: Describe at least one example of a time in lab today when you
%   needed to go beyond the lab instructions and instructor’s help to solve a problem or
%   figure out your next step. How were you able to do that?
    \title{Lab 2}
    \maketitle
    \begin{enumerate}
        \item I learned that capacitors charge in an exponential fashion, and that LED's have a completely flat
        response when reverse biased (at least at the voltage ranges we were looking at).
        I also learned that when the capacitor changes from the high frequency to the low frequency regime it behaves
        strangely and introduces a phase shift in between it and the next circuit component.
        \item I learned a lot about oscilloscopes.
        There was a lot I learned simply with navigating through the settings on the machine, which also helped me
        better understand what I need to be looking for when analyzing a signal.
        For example, the fact that you can easily change the vertical $V_0$ offset, and that you change the
        time-scale to shift it horizontally I think gives me a slightly better understanding of AC signals.
        I have also never used breadboards before, and now I understand at least how they are structured, and could
        probably identify what is going on if I was shown a circuit.
        \item I noticed that we had a BNC splitter, and I used that to look at the signal from the signal generator
        and the signal from the circuit component simultaneously.
        This helped me more easily see the voltage drop across different components, and be able to see the phase
        shift between the signal generator and the voltage drop across the resistor.
    \end{enumerate}


\end{document}