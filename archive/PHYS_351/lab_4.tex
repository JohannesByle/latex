\documentclass[12pt]{article}

\usepackage{amsmath}

\author{Johannes Byle}
\begin{document}
%    Learning Outcome 1: What specifically did you learn about circuits or circuit
%   components (resistors, capacitors, inductors, diodes, transistors, op amps) that you did
%   not know coming into lab? Identify at least two examples.
%   2. Learning Outcome 2: What did you learn about one or more of the tools and
%   techniques listed in Learning Outcome 2 that you did not know coming into lab?
%   Identify at least two examples.
%   3. Learning Outcome 3: Describe at least one example of a time in lab today when you
%   needed to go beyond the lab instructions and instructor’s help to solve a problem or
%   figure out your next step. How were you able to do that?
%    Develop competency with several commonly used electronics tools/techniques:
%o Digital MultiMeter (DMM)
%o Digital Oscilloscope
%o Prototyping Breadboard
%o Signal Generator
%o Soldering
%o Circuit Simulation (with Spice)
%o Debugging Circuit Problems

    \title{Lab 4}
    \maketitle
    \begin{enumerate}
        \item I learned about op amps and what they are and how they work.
        I did not know how what op amps or even active filters are, but now I understand that there are filters that
        can amplify a signal by accepting power from a different power source.
        I also learned about op amps, and that they are a component that can amplify an input voltage based on the
        resistance across their connections.
        \item I learned about what a proper breadboard connection should feel like, as I had a resistor slide along
        the bottom and not make a proper connection.
        I also learned more about using the auto zoom feature on the oscilloscope, what I found most helpful in this
        lab was only having horizontal auto zoom, and adjusting the vertical zoom manually.
        \item After being completely stuck with a weird problem where no matter what I did the gain was 1, I
        completely redid the circuit, and tried reducing the number of wires.
        This initially did not work, but it gave me a different output than before.
        After messing around with the connections I noticed one of the resistors slid in further than it should, and
        when I moved it to a different pin on the ground rail I started getting a normal output.
    \end{enumerate}


\end{document}