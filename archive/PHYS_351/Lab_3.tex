\documentclass[12pt]{article}

\usepackage{amsmath}

\author{Johannes Byle}
\begin{document}
%    Learning Outcome 1: What specifically did you learn about circuits or circuit
%   components (resistors, capacitors, inductors, diodes, transistors, op amps) that you did
%   not know coming into lab? Identify at least two examples.
%   2. Learning Outcome 2: What did you learn about one or more of the tools and
%   techniques listed in Learning Outcome 2 that you did not know coming into lab?
%   Identify at least two examples.
%   3. Learning Outcome 3: Describe at least one example of a time in lab today when you
%   needed to go beyond the lab instructions and instructor’s help to solve a problem or
%   figure out your next step. How were you able to do that?
    \title{Lab 3}
    \maketitle
    \begin{enumerate}
        \item I learned about how transistors work.
        I didn't know anything about transistors before this lab, or at least I must have forgotten what I had learned.
        But now I know that they are basically like electronic dimmer switches.
        I also now realize that because larger capacitors take longer to charge, if they will oscillate they will
        probably oscillate at a lower frequency.
        \item Although I didn't learn anything completely new about breadboards, I have gotten a lot better at
        understanding how the they work.
        At the beginning of the lab I just blindly followed the instructions, but by the end I was able to see what
        my connections were doing, and try to imagine what the circuit diagram would look like.
        I think having to design my own section of the circuit really helped with that, I was able to see which
        components were in parallel, an which were in series just by looking at how they were connected.
        \item I learned pretty quickly that simply bumping different components when my circuit wasn't working could
        help me identify components that weren't connected properly.
    \end{enumerate}


\end{document}