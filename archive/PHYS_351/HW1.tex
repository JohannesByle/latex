\documentclass[12pt]{article}
\usepackage{amsmath}
\usepackage{circuitikz}
\usepackage{xcolor}

\author{Johannes Byle}
\title{HW Set 1}
\begin{document}
    \maketitle
    \noindent
    \textbf{Question 1}\\
    \resizebox{0.5\textwidth}{!}{%
    \begin{circuitikz}
        \draw (0,0) to[battery, l_=10kV] (10,0)
        to [R, l_=1M$\Omega$] (10, -1.5)
        to [R, l_=1M$\Omega$] (10, -3)
        to [R, l_=1M$\Omega$] (10, -4.5)
        to [R, l_=1M$\Omega$] (10, -6)
        to [R, l_=1M$\Omega$] (10, -7.5)
        to [R, l_=1M$\Omega$] (10, -9)
        to [R, l_=1M$\Omega$] (10, -10.5)
        to [R, l_=1M$\Omega$] (10, -12)
        to [R, l_=1M$\Omega$] (10, -13.5)
        to [R, l_=1M$\Omega$] (10, -15);
        \draw (10, -15) to (7.5, -15)
        to [R, l_=1k$\Omega$] (6, -15)
        to [R, l_=1k$\Omega$] (4.5, -15)
        to [R, l_=1k$\Omega$] (3, -15)
        to [R, l_=1k$\Omega$] (1.5, -15)
        to [R, l_=1k$\Omega$] (0, -15);
        \draw (0, 0) to (0, -17.5)
        to [short, -o] (3.5, -17.5);
        \draw (10, -15) to (10, -17.5)
        to [short, -o] (6.5, -17.5);
    \end{circuitikz}
    }
    \\
    Assuming the data monitoring system can be modeled as a short, the total current across the circuit is
    $I=\frac{10\text{ kV}}{10\text{ M}\Omega+5\text{ k}\Omega}\approx10^{-5}\text{ A}$.
    Thus the maximum power dissipated by any single resistor cannot be more than
    $P=I^{2}R=(10^{-5})^2\text{ A}\times1\text{ M}\Omega\approx1\text{ W}$\\

    \noindent
    \textbf{Question 2}\\
    $-16=-2I_{1} - 2(I_{1}+I_{2}) - I_{1}+I_{3}$\\
    $-9=-2(I_{2}+I_{3}) - 2(I_{2}+I_{1}) - I_{2}$\\
    $0=-4I_{3}-I_{3}+I_{1}-2(I_{3}+I_{2})$\\
    $\begin{bmatrix}
         -5 & -2 & 1\\
         -2 & -5 & -2\\
         1 & -2 & -7
    \end{bmatrix}^{-1}
    \begin{bmatrix}
        -16\\
        -9\\
        0
    \end{bmatrix}\approx
    \begin{bmatrix}
        3.1\\
        0.4\\
        0.3
    \end{bmatrix}$\\

    \noindent
    \textbf{Question 3}\\
    $-16=-r_1I_{1} - r_3(I_{1}+I_{2}) - L_4(\frac{\delta I_{1}}{\delta t}+\frac{\delta I_{3}}{\delta t})$\\
    $-9=-r_5(I_{2}+I_{3}) - r_3(I_{2}+I_{1}) - L_2\frac{\delta I_{2}}{\delta t}$\\
    $0=-r_6I_{3}-r_4(I_{3}-I_{1})-r_5(I_{3}+I_{2})$\\

\end{document}