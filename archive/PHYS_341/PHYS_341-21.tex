\documentclass[english]{article}
\usepackage[T1]{fontenc}
\usepackage[latin9]{inputenc}
\usepackage{babel}
\usepackage{graphicx}
\usepackage{caption}
\usepackage{subcaption}
\usepackage{float}
\usepackage{amsmath}
\usepackage[dvipsnames]{xcolor}
\usepackage{gensymb}
\usepackage{pgfplots}
\usepackage{multicol}
\begin{document}
\begin{multicols*}{2}
\textbf{Johannes Byle}\\
\newcommand{\Lagr}{\mathcal{L}}

\noindent
\textbf{9.15}
From equation $9.53$ we know that the radial acceleration is:
$$\ddot{z}=-g+2\Omega\dot{x}\sin\theta$$
At the north pole $\theta$ is 0
$$\ddot{z}=-g=g_0$$
At the equator $\theta$ is $\pi$
$$\ddot{z}=-g+2\Omega\dot{x}=g_0(1+\frac{2\Omega\dot{x}}{g_0})=\lambda g_0$$

\noindent
\textbf{9.20 (a)}
$$m\ddot{r}=F+2m\dot{r}\times\Omega+m(\Omega\times r)\times\Omega$$
Since $r=x\hat{x}+y\hat{y}$
$$m\ddot{x}=F_x+2m\dot{y}\Omega+m\Omega^2x$$
$$m\ddot{y}=F_y+2m\dot{x}\Omega+m\Omega^2y$$
\textbf{(b)}
$$\eta=x+iy$$
$$\ddot{\eta}=2i\Omega\dot{\eta}+\Omega^2\eta$$
$$\eta=e^{-i\alpha t}$$
$$\dot{\eta}=-i\alpha e^{-i\alpha t}$$
$$\ddot{\eta}=-\alpha^2 e^{-i\alpha t}$$
$$-\alpha^2=2\Omega\alpha+\Omega^2$$
$$\alpha=\Omega$$
Since this is only one solution we need to guess $te^{-i\alpha t}$
$$\eta=e^{-i\Omega t}(A+Bt)$$
\textbf{(c)}
$$\eta(0)=e^{-i\Omega (0)}(A+B(0))=A=x_0$$
$$\dot{\eta}(0)=-i\Omega e^{-i\Omega (0)}(x_0+B(0))+Be^{-i\Omega (0)}$$
$$\dot{\eta}(0)=B-\Omega x_0i=v_{x0}+v_{y0}i$$
$$B=v_{x0}+(v_{y0}+\Omega x_0)i$$
$$e^{it}=\cos t+i\sin t$$
$$x(t)=(x_0+v_{x0}t)\cos\Omega t+(v_{y0}+\Omega x_0)t\sin\Omega t$$
$$y(t)=-(x_0+v_{x0}t)\sin\Omega t+(v_{y0}+\Omega x_0)t\cos\Omega t$$
\textbf{(d)}
It moves in a spiral

\noindent
\textbf{9.22}
$$m\ddot{r}=q(Q+\dot{r}\times B)$$
$$m\ddot{r}=q(Q+\dot{r}\times B)+2m\dot{r}\times\Omega+m(\Omega\times r)\times\Omega$$
Which is an ellipse

\end{multicols*}
\end{document}