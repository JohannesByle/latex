\documentclass[english]{article}
\usepackage[T1]{fontenc}
\usepackage[latin9]{inputenc}
\usepackage{babel}
\usepackage{graphicx}
\usepackage{caption}
\usepackage{subcaption}
\usepackage{float}
\usepackage{amsmath}
\usepackage[dvipsnames]{xcolor}
\usepackage{gensymb}
\begin{document}
\textbf{HW Jan 23, Johannes Byle}\\

\noindent
\textbf{1.39} $F=ma$ stays the same. The position of the ball is $(v_{0}cos\theta t-\frac{1}{2}gsin\phi t^2,(v_{0}sin\theta t-\frac{1}{2}gcos\phi t^2,0)$. Time it takes for the ball to reach the final position is where $0=t(2v_{0}sin\theta-gsin\phi t)$ or $t=\frac{2v_{0}sin\theta}{gsin\phi}$. Thus the ball is a distance of $\frac{2v_0^2cos\phi sin\theta}{gsin\phi}=\frac{2v_0^2cos(\phi+\theta) sin\theta}{gcos^2\phi}$ away from the origin. The maximum upward range is where $x=v_{0}\frac{v_{0}}{gsin\phi}-\frac{1}{2}gsin\phi (\frac{v_{0}}{gsin\phi})^2$ or $x=\frac{v_0^2}{g(1+sin\phi)}$.\\

\textbf{1.41} Equation 1.48 is $F_r=m(\ddot{r}-r\dot{\phi}^2),F_{\phi}=m(r\ddot{\phi}+2\dot{r}\dot{\phi})$ Since R is constant $T=mR\omega^2$.\\

\textbf{1.43 (a)} $\hat{r}=\frac{x\hat{x}+y\hat{y}}{\sqrt{x^2+y^2}}$, and $\sqrt{x^2+y^2}=\sqrt{(rsin\theta cos\phi)^2+(rsin\theta sin\phi)^2}$, or $\sqrt{x^2+y^2}=\sqrt{r^2sin^2\theta(cos^2\phi+sin^2\phi)}=rsin\theta$. Thus $\hat{r}=\frac{rsin\theta cos\phi\hat{x}+rsin\theta sin\phi\hat{y}}{rsin\theta}=\hat{x}cos\phi+\hat{y}sin\phi$. We know $\hat{\phi}=-sin\phi\hat{x}+cos\phi\hat{y}$.\\


\textbf{(b)} $\dot{\hat{r}}=-sin\phi\dot{\phi}\hat{x}+cos\phi\dot{\phi}\hat{y}$ and $\dot{\hat{\phi}}=-cos\phi\dot{\phi}\hat{x}-sin\phi\dot{\phi}\hat{y}$

\end{document}