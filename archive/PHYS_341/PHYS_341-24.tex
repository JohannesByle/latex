\documentclass[english]{article}
\usepackage[T1]{fontenc}
\usepackage[latin9]{inputenc}
\usepackage{babel}
\usepackage{graphicx}
\usepackage{caption}
\usepackage{subcaption}
\usepackage{float}
\usepackage{amsmath}
\usepackage[dvipsnames]{xcolor}
\usepackage{gensymb}
\usepackage{pgfplots}
\usepackage{multicol}
\begin{document}
\begin{multicols*}{2}
\textbf{Johannes Byle}\\
\newcommand{\Lagr}{\mathcal{L}}

\noindent
\textbf{10.30}
$$I_{xx}=\rho\int y^2dy$$
$$I_{xy}=\rho\int\int xydxdy$$
$$I_{xz}=0$$
$$I_{yx}=\rho\int\int xydxdy$$
$$I_{yy}=\rho\int x^2dx$$
$$I_{yz}=0$$
$$I_{zx}=0$$
$$I_{zy}=0$$
$$I_{zz}=\rho\int\int x^2+y^2dxdy$$
$I=
\begin{bmatrix}
  \rho\int y^2dy & \rho\int\int xydxdy & 0 \\
  \rho\int\int xydxdy & \rho\int x^2dx & 0 \\
  0 & 0 & \rho\int\int x^2+y^2dxdy
 \end{bmatrix}
$
$$\det I= \rho^2\int\int x^2+y^2dxdy\Big[\rho\int y^2dy\int x^2dx-\Big(\int\int xydxdy\Big)^2\Big]$$
Thus the perpendicular axis is perpendicular to the lamina.

\noindent
\textbf{10.35(a)}\\
$I=ma^2
\begin{bmatrix}
  17 & 0 & 0 \\
  0 & 6 & -1 \\
  0 & -1 & 6
 \end{bmatrix}
$\\
\textbf{(b)}\\
$$\det(I-\lambda I_3)=5,7,17$$
$$v_1=(0,1,1)$$
$$v_2=(0,-1,1)$$
$$v_3=(1,0,0)$$

\noindent
\textbf{10.39}\\
$$\Omega=\frac{MgR}{\lambda\omega}\hat{z}=\frac{MgR}{\frac{3}{10}Mr^2\omega}\hat{z}$$
$$\Omega=\frac{9.8\times10}{\frac{3}{10}2.5^2\times1800}\approx0.029$$
\end{multicols*}
\end{document}