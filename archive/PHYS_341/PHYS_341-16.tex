\documentclass[english]{article}
\usepackage[T1]{fontenc}
\usepackage[latin9]{inputenc}
\usepackage{babel}
\usepackage{graphicx}
\usepackage{caption}
\usepackage{subcaption}
\usepackage{float}
\usepackage{amsmath}
\usepackage[dvipsnames]{xcolor}
\usepackage{gensymb}
\usepackage{pgfplots}
\usepackage{multicol}
\begin{document}
\begin{multicols*}{2}
\textbf{HW March 4, Johannes Byle}\\
\newcommand{\Lagr}{\mathcal{L}}

\noindent
\textbf{8.2 (a)}
$$\Lagr=T-U=\frac{1}{2}M\dot{R}^2+\frac{1}{2}\mu\dot{r}^2-U(r)-MgR$$
$$\Lagr=\Lagr_{cm}+\Lagr_{rel}$$

\noindent
\textbf{(b)}
$$\Lagr=\frac{1}{2}M\dot{X}^2$$
$$\Lagr=\frac{1}{2}M\dot{Y}^2-MgY$$
$$\Lagr=\frac{1}{2}M\dot{Z}^2$$
The center of mass moves the same as without the gravitational force, except it has a constant downward acceleration in one dimension. 
$$\Lagr=\frac{1}{2}\mu\dot{x}^2-U(x)$$
$$\Lagr=\frac{1}{2}\mu\dot{y}^2-U(y)$$
$$\Lagr=\frac{1}{2}\mu\dot{z}^2-U(z)$$
Which is the same as 
$$\Lagr=\frac{1}{2}\mu\dot{r}^2-U(r)$$
\end{multicols*}
\end{document}