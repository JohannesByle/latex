\documentclass{article}
\begin{document}
\title{Literature Review}
\author{Johannes Byle}
\maketitle

%Last Name, First Name. “Article Title.” Journal Name Volume Number (Year Published): Page Numbers.

\subsection*{Analysis of Mechanics in Jenga}
Ziglar, James. “Analysis of Mechanics in Jenga” Robotics Institute, Carnegie Mellon University.\\

Although Jenga pieces are square this article had useful information about the forces and the torques on stacked objects. The article did not describe the pitching of the blocks, as it would not be a smart Jenga move. However, I believe for tall cylinder sections there is a real possibility of pitching, and that is what I think will be most difficult to describe.

\subsection*{Fun with stacking blocks}
Hall, John. “Fun with stacking blocks” American Journal of Physics 73 (2005).\\

This article was a lot more in depth than the previous one, and it included a lot more details on moment of inertia and torque. However, it was all about rectangular blocks and thus has the same limitations as the previous article.

\subsection*{Simulating Object Stacking Using Stack Stability}
Thomsen, Kasper. “Simulating Object Stacking Using Stack Stability” Aalborg University Department of Architecture (2014).\\

Even after searching for stacked pillars, stacked cylinders and column buckling I couldn't find any articles on stacking cylinders specifically. Despite still being about blocks this article went further in depth about leaning blocks and other rotations problems, and was thus still helpful.


\end{document}