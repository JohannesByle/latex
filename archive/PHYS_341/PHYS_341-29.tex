\documentclass[english]{article}
\usepackage[T1]{fontenc}
\usepackage[latin9]{inputenc}
\usepackage{babel}
\usepackage{graphicx}
\usepackage{caption}
\usepackage{subcaption}
\usepackage{float}
\usepackage{amsmath}
\usepackage[dvipsnames]{xcolor}
\usepackage{gensymb}
\usepackage{pgfplots}
\usepackage{multicol}
\begin{document}
\begin{multicols*}{2}
\textbf{Johannes Byle}\\
\newcommand{\Lagr}{\mathcal{L}}

\noindent
\textbf{15.1}\\
Newtons first law can be expressed as:
$$\sum F=0\Leftrightarrow\frac{dv}{dt}=0$$
Assuming mass is a nonzero constant.
Under Galilean transformation:
$$v'=v+\beta$$
Since the derivative of this is still zero the first law still holds.
$$\frac{d}{dt}(v+\beta)=0$$
Newtons third law can be written as:
$$F_a=-F_b$$
$$m_a\frac{dv_a}{dt}=m_b\frac{dv_b}{dt}$$
Under Galilean transformation both derivatives would not change for the same reason as stated above:
$$\frac{d}{dt}(v+\beta)=\frac{d}{dt}(v)+\frac{d}{dt}(\beta)=\frac{dv}{dt}$$
\noindent
\textbf{15.6}\\
$$t'=\frac{t}{\sqrt{1-\frac{v^2}{c^2}}}$$
$$\frac{t}{t'}=\sqrt{1-\frac{v^2}{c^2}}$$
$$c\sqrt{1-\frac{t^2}{t'^2}}=v$$
$$v=\sqrt{\frac{2}{3}}c$$
\end{multicols*}
\end{document}