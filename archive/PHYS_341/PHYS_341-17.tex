\documentclass[english]{article}
\usepackage[T1]{fontenc}
\usepackage[latin9]{inputenc}
\usepackage{babel}
\usepackage{graphicx}
\usepackage{caption}
\usepackage{subcaption}
\usepackage{float}
\usepackage{amsmath}
\usepackage[dvipsnames]{xcolor}
\usepackage{gensymb}
\usepackage{pgfplots}
\usepackage{multicol}
\begin{document}
\begin{multicols*}{2}
\textbf{HW March 4, Johannes Byle}\\
\newcommand{\Lagr}{\mathcal{L}}

\noindent
\textbf{8.4}
$$\Lagr=\frac{1}{2}\mu\dot{x}^2-U(x)$$
$$\Lagr=\frac{1}{2}\mu\dot{y}^2-U(y)$$
$$\Lagr=\frac{1}{2}\mu\dot{z}^2-U(z)$$
The motion of a single particle in three dimension would be:
$$\Lagr=\frac{1}{2}\mu(\dot{x}^2+\dot{y}^2+\dot{z}^2)-U(x,y,z)$$
Since $r=\sqrt{x^2+y^2+z^2}$
$$\Lagr=\frac{1}{2}\mu(\dot{x}^2+\dot{y}^2+\dot{z}^2)-U(x,y,z)=\frac{1}{2}\mu\dot{r}-U(r)$$

\noindent
\textbf{8.12 (a)}
$$U_{eff}(r)=-\frac{Gm_1m_2}{r}+\frac{l^2}{2\mu r^2}$$	
$$\frac{dU_{eff}}{dr}=\frac{Gm_1m_2}{r^2}-\frac{l^2}{\mu r^3}$$
$$0=r^2\Big(\frac{1}{Gm_1m_2}-\frac{r}{l^2}\big)$$
It will orbit at:
$$r=\frac{l^2}{Gm_1m_2}$$
\textbf{(b)} $$\frac{d^2U_{eff}}{dr^2}=-\frac{2Gm_1m_2}{r^3}+\frac{3l^2}{\mu r^4}$$
$$\frac{d^2U_{eff}}{dr^2}=-\frac{2Gm_1m_2}{(\frac{l^2}{Gm_1m_2})^3}+\frac{3l^2}{\mu (\frac{l^2}{Gm_1m_2})^4}$$
$$\frac{d^2U_{eff}}{dr^2}=-\frac{2(Gm_1m_2)^4}{l^6}+\frac{3(Gm_1m_2)^4l^2}{\mu l^8}$$
$$\frac{d^2U_{eff}}{dr^2}=-\frac{2(Gm_1m_2)^4}{l^6}+\frac{3(Gm_1m_2)^4l^2}{\mu l^8}$$
$$\frac{d^2U_{eff}}{dr^2}=\frac{(Gm_1m_2)^4}{l^6}\Big(\frac{3}{\mu}-2\Big)$$
Whether or it is in a stable orbit is dependent on the value of $\mu$ 
\end{multicols*}
\end{document}