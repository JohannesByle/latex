\documentclass[english]{article}
\usepackage[T1]{fontenc}
\usepackage[latin9]{inputenc}
\usepackage{babel}
\usepackage{graphicx}
\usepackage{caption}
\usepackage{subcaption}
\usepackage{float}
\usepackage{amsmath}
\usepackage[dvipsnames]{xcolor}
\usepackage{gensymb}
\begin{document}
\textbf{HW Jan 23, Johannes Byle123}\\

\noindent
\textbf{2.4 (a)} The amount of mass that is directly in contact with the projectile is the area of the projectile times the density of the fluid $\rho A$. Since the mass density of the fluid is constant in any direction the amount of mass encountered of a period of time is $\rho Av$.\\

\textbf{(b)} In this example $\frac{\Delta E}{\Delta x}=F$. The amount of energy change per unit time is is $\frac{1}{2}\rho Av^3$. To get the force we divide by $x$, making the drag force per unit time $F=\frac{1}{2}\frac{\rho Av^2}{t}$. \\

\textbf{(c)} $f_{quad}=\gamma D^2v^2=k\rho \frac{\pi D^2}{4}v^2$ thus $\gamma=k\rho \frac{\pi}{4}\approx 0.25$\\

\noindent
\textbf{2.6 (a)} We know $v_y(t)=v_{ter}(1-e^{-t/\tau})$, we also know $v_{ter}=g\tau$. Thus if Taylor approximations are good at low values of $x$ then $v_y(t)=g\tau(1-(1+(-t/\tau)))=gt$.\\

\textbf{(b)} We know $y(t)=v_{ter}t+(v_{y0}-v_{ter})\tau(1-e^{-t/\tau})$, from part (a) we can reduce this to $y(t)=g\tau t+(v_{y0}-g\tau)\tau(1-(1+(-t/\tau)))$ which is $\frac{1}{2}gt^2$.\\

\noindent
\textbf{2.11 (a)} $v_y(t)=v_0-v_{ter}(1-e^{-t/\tau})-gt$\\

\textbf{(b)} The highest point would be where $v_0=v_{ter}(1-e^{-t/\tau})+gt$. The position at this point would be $y_{max}(t)=v_0-v_{ter}t+(v_{y0}-v_{ter})\tau(1-e^{-t/\tau})-\frac{1}{2}gt^2$\\

\noindent
\textbf{2.31 (a)} $v_{ter}=\frac{mg}{b}$ and $b=\beta D$, thus $v_{ter}=20.2\frac{m}{s}$.\\

\textbf{(b)} In vacuum it would take $\sqrt{\frac{60}{9}}$.
\end{document}