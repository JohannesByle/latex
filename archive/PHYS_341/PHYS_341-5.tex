\documentclass[english]{article}
\usepackage[T1]{fontenc}
\usepackage[latin9]{inputenc}
\usepackage{babel}
\usepackage{graphicx}
\usepackage{caption}
\usepackage{subcaption}
\usepackage{float}
\usepackage{amsmath}
\usepackage[dvipsnames]{xcolor}
\usepackage{gensymb}
\begin{document}
\textbf{HW Jan 28, Johannes Byle}\\

\noindent
\textbf{2.46 (a)} Real = $1$, Imaginary = $1$, Modulus = $\sqrt{2}$, Phase = $\frac{\pi}{4}$, $z^*$ = $1-i$\\

\textbf{(b)} Real = $1$, Imaginary = $-\sqrt{3}$, Modulus = $2$, Phase = $-\frac{\pi}{3}$, $z^*$ = $1+i\sqrt{3}$\\ 

\textbf{(c)} Real = $1$, Imaginary = $-1$, Modulus = $\sqrt{2}$, Phase = $-\frac{\pi}{4}$, $z^*$ = $1-i$\\

\textbf{(d)} Real = $5cos(\omega t)$, Imaginary = $5sin(\omega t)$, Modulus = $5$, Phase = $\omega t$, $z^*$ = $5(cos(\omega t)-sin(\omega)t)$\\

\noindent
\textbf{2.47 (a)} \textbf{z+w} = $9+4i$, \textbf{z-w} = $3+12i$, \textbf{zw} = $50$, \textbf{z/w} = $-\frac{14}{25}+\frac{48}{25}i$\\

\textbf{(b)} \textbf{z+w} = $8cos\frac{\pi}{3}+4cos\frac{\pi}{6}+i(8sin\frac{\pi}{3}+4sin\frac{\pi}{6})$, \textbf{z-w} = $8cos\frac{\pi}{3}+4cos\frac{\pi}{6}-i(8sin\frac{\pi}{3}+4sin\frac{\pi}{6})$, \textbf{zw} = $32e^{i\frac{\pi}{2}}$, \textbf{z/w} = $2e^{i\frac{\pi}{6}}$\\

\noindent
\textbf{2.52} $\eta=ae^{i(\delta-\omega t)}$, thus $v_x=acos(\delta-\omega t)$ and  $v_y=asin(\delta-\omega t)i$. This means that both $v_x$ and $v_y$ oscillate over time at a rate of $\omega t$.

\noindent
\textbf{3.2} The other fragment would be traveling south with with horizontal velocity $v_0$ and straight downwards with vertical velocity $v_0$.
\end{document}