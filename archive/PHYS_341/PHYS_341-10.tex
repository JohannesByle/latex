\documentclass[english]{article}
\usepackage[T1]{fontenc}
\usepackage[latin9]{inputenc}
\usepackage{babel}
\usepackage{graphicx}
\usepackage{caption}
\usepackage{subcaption}
\usepackage{float}
\usepackage{amsmath}
\usepackage[dvipsnames]{xcolor}
\usepackage{gensymb}
\begin{document}
\textbf{HW Feb 12, Johannes Byle}\\

\noindent
\textbf{4.53 (a)} $$\frac{ke^2}{r^2}=\frac{mv^2}{r}$$
$$\frac{ke^2}{r}=mv^2$$
$$KE=\frac{1}{2}mv^2=\frac{ke^2}{2r}$$
$$PE=\int \frac{ke^2}{r^2}dr=-\frac{ke^2}{r}$$
Thus
$$\frac{1}{2}PE=-KE$$
\textbf{(b)} $$E_{e_1}=PE_1+KE_1=\frac{1}{2}mv^2-\frac{ke^2}{r}$$
$$E_{e_2}=PE_2+KE_2=T_2-\frac{ke^2}{r}$$
$$E_p=PE_3=-\frac{ke^2}{r}$$
\textbf{(c)} Before:\\ 
$$E_{e_1}=PE_1+KE_1=\frac{1}{2}mv^2-\frac{ke^2}{r}$$
$$E_{e_2}=PE_2+KE_2=T_2$$
$$E_p=PE_3=-\frac{ke^2}{r}$$
After:\\
$$E_{e_1}=PE_1+KE_1=\frac{1}{2}mv^2$$
$$E_{e_2}=PE_2+KE_2=\frac{1}{2}mv^2-\frac{ke^2}{r'}$$
$$E_p=PE_3=-\frac{ke^2}{r}$$
\end{document}