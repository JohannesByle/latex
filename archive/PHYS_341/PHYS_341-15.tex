\documentclass[english]{article}
\usepackage[T1]{fontenc}
\usepackage[latin9]{inputenc}
\usepackage{babel}
\usepackage{graphicx}
\usepackage{caption}
\usepackage{subcaption}
\usepackage{float}
\usepackage{amsmath}
\usepackage[dvipsnames]{xcolor}
\usepackage{gensymb}
\usepackage{pgfplots}
\usepackage{multicol}
\begin{document}
\begin{multicols*}{2}
\textbf{HW March 1, Johannes Byle}\\
\newcommand{\Lagr}{\mathcal{L}}

\noindent
\textbf{7.8(a)} $$T=\frac{1}{2}m_1\dot{x_1}^2+\frac{1}{2}m_2\dot{x_2}^2$$
$$U=\frac{1}{2}k(x_1-x_2-l)^2$$
$$\Lagr=\frac{1}{2}m_1\dot{x_1}^2+\frac{1}{2}m_2\dot{x_2}^2-\frac{1}{2}k(x_1-x_2-l)^2$$

\textbf{(b)} $$\Lagr=2m\dot{X}^2-\frac{1}{2}kx^2$$

\textbf{(c)} $$\frac{\delta\Lagr}{\delta X}=0$$
$$\frac{\delta\Lagr}{\delta \dot{X}}=4m\dot{X}=C$$
$$\frac{\delta\Lagr}{\delta x}=-kx=\ddot{x}$$
$$X(t)=\frac{C}{4m}t$$
$$x(t)=Asin(kt)$$

\noindent
\textbf{7.12} $$\Lagr=T-U$$
In order to add $F_{fric}$ to the above equation we would have to take the integral of the friction force and subtract it from the U term. When we convert the Lagrangian to the derivative form we then convert $F_{fric}$ back to a basic force form, leaving the equation in the book.

\noindent
\textbf{7.23} $$\Lagr=T-U$$
$$T=\frac{1}{2}m(\dot{X}+\dot{x})^2+\frac{1}{2}M\dot{X}^2$$
$$U=\frac{1}{2}kx^2$$
$$\frac{\delta\Lagr}{\delta X}=0$$
$$\frac{\delta\Lagr}{\delta \dot{X}}=m(\dot{X}+\dot{x})+M\dot{X}=C$$
$$\frac{\delta\Lagr}{\delta x}=-kx=\frac{d}{dt}m(\dot{X}+\dot{x})$$
$$\ddot{X}=-\omega^2A\cos(\omega t)$$
$$-kx=m\ddot{X}+m\ddot{x}$$
$$\ddot{x}+\omega_0^2x=B\cos\omega t$$
\end{multicols*}
\end{document}