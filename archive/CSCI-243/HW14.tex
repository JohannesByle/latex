\documentclass[12pt]{article}
\usepackage[utf8]{inputenc}
\usepackage{amsmath}
\usepackage{amssymb}
\usepackage{tikz}
\usepackage{venndiagram}
\usepackage{enumitem}
\usepackage{amsfonts}
\usetikzlibrary{patterns}


\title{HW 14}
\author{Johannes Byle}
\newcommand{\pset}[1]{$$\mathbb{P}(#1)}

\begin{document}
    % 4.9.(1, 3, 4, 6)
    \maketitle
    \noindent
    \textbf{4.9.1}\\
%    \textbf{Proof.} Suppose $B\subseteq A$.
%    Further suppose $X\in \mathbb{P}(B)-\mathbb{P}(A)$.\\
%    By definition of difference, $X\in\mathbb{P}(B)$ and $X\notin\mathbb{P}(A)$.
%    By definition of powerset, $X\subseteq B$ and $X\subseteq A$.\\
%    Suppose $X\in X$.
%    By definition of subset, $x\in B$.
%    By definition of subset agains, $x\in A$.
%    By definition subset yet again (this time synthetically), $X\subseteq A$.
%    Contradiction.
%    Therefore $X\in\mathbb{P}(B)-\mathbb{P}(A)=\emptyset$ $\square$\\
    \textbf{Proof.} Suppose $A\subseteq B$.
    Further suppose $X\in\mathbb{P}(A)$.
    Then, by definition of powerset $X\subseteq A$.
    By definition of subset $X\subseteq B$.\\
    Then by definition of powerset $X\in\mathbb{P}(B)$, therefore by definition of subset $\mathbb{P}(A)\subseteq\mathbb{P}(B)$ $\square$.\\


    \maketitle
    \noindent
    \textbf{4.9.3}\\
    \textbf{Proof.} Suppose $X\in\mathbb{P}(A)\cap\mathbb{P}(B)$.
    Then, by definition of intersect $X\in\mathbb{P}(A)$ and $X\in\pset{B}$.
    By definition of powerset $X\subseteq A$.
    By definition of powerset $X\subseteq B$.
    Then, by definition of intersect $X\subseteq A\cap B$.\\
    Therefore by definition of powerset $X\in\pset{A\cap B}$ $\square$.\\

    \maketitle
    \noindent
    \textbf{4.9.4}\\
    \textbf{Proof.} Suppose $B\in\pset{A-C}$.
    Further suppose $x\in B$.
    By definition of powerset $B\subseteq A-C$.
    Then, by definition of subset $x\in A-C$
    Then, by definition of difference $x\in A$ and $x\notin C$.
    Then, by definition of subset $B\subseteq A$.\\
    Therefore, by definition of powerset $B\in\pset{A}$ $\square$.\\

    \maketitle
    \noindent
    \textbf{4.9.6}\\
    \textbf{Proof.} Suppose $a\in A$.
    Further suppose $X\in\pset{A-\{a\}}\cap\{C\cup\{a\}|C\in \pset{A-\{a\}}\}$.
    By definition of intersect $X\in\pset{A-\{a\}}$ and $X\in\{C\cup\{a\}|C\in \pset{A-\{a\}}\}$
    By definition of powerset $X\subseteq A-\{a\}$.
    Suppose $x\in X$.
    By definition of subset $x\in A-\{a\}$
    Then, by definition of difference, $X\in A$ and $x\notin \{a\}$.
    By definition of union $a\in X\forall X\in\{C\cup\{a\}|C\in \pset{A-\{a\}}\}$, therefore $X\in\pset{A-\{a\}}\cap\{C\cup\{a\}|C\in \pset{A-\{a\}}\}=\emptysetw.

\end{document}