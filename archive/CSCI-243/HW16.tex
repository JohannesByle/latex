\documentclass[12pt]{article}
\usepackage[utf8]{inputenc}
\usepackage{amsmath}
\usepackage{amssymb}
\usepackage{tikz}
\usepackage{venndiagram}
\usepackage{enumitem}
\usepackage{amsfonts}
\usetikzlibrary{patterns}


\title{HW 16}
\author{Johannes Byle}
\newcommand{\pset}[1]{$$\mathbb{P}(#1)}

\begin{document}

    \maketitle
    \noindent
    \textbf{5.3.8}\\
    $\mathcal{I}_{R^{-1}}(\mathcal{I}_R(A))\subseteq A$ is false\\
    \textbf{Proof}. Counter example $X=\{x, a\}$, $Y=\{y\}$, $R=\{(x, y), (a, y)\}$ and $A=\{a\}$.\\
    $\mathcal{I}_R(A)=\{y\}$\\
    $\mathcal{I}_{R^{-1}}(\{y\})=\{x, a\}$\\
    $\{x, a\}\nsubseteq\{a\}$\\

    \smallskip
    \noindent
    \textbf{Proof.} Suppose $a\in A$.
    By definition of subset $a\in X$, then by definition of relation and by definition of image
    $Y\subseteq\mathcal{I}_{R}(A)$.
    Then by definition of image and by definition of relation $X\subseteq\mathcal{I}_{R^{-1}}(\mathcal{I}_R(A))$.
    By definition of subset $a\in\mathcal{I}_{R^{-1}}(\mathcal{I}_R(A))$, therefore $A\subseteq \mathcal{I}_{R^{-1}}
    (\mathcal{I}_R(A))$.

    \medskip
    \noindent
    \textbf{5.3.10}\\
    \textbf{Proof.} Suppose $(a, b)\in R$.
    By definition of inverse $(b, a)\in R^{-1}$, by definition again $(a, b)\in (R^{-1})^{-1}$.
    Therefore $R \subseteq (R^{-1})^{-1}$.
    Further suppose $(a, b)\in (R^{-1})^{-1}$.
    By definition of inverse $(b, a)\in R^{-1}$, by definition again $(a, b)\in R$.
    Therefore $(R^{-1})^{-1}\subseteq R$.
    Therefore, by definition of subset $(R^{-1})^{-1}= R$

\end{document}