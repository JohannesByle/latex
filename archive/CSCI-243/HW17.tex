\documentclass[12pt]{article}
\usepackage[utf8]{inputenc}
\usepackage{amsmath}
\usepackage{amssymb}
\usepackage{tikz}
\usepackage{venndiagram}
\usepackage{enumitem}
\usepackage{amsfonts}
\usetikzlibrary{patterns}


\title{HW 17}
\author{Johannes Byle}
\newcommand{\image}[2]{\mathcal{I}_{#1}(#2)}

\begin{document}

    \maketitle
    \noindent
    \textbf{5.3.5}\\
    \textbf{Proof.} Suppose $R$ is a relation from $X$ to $Y$, $S$ is a relation from $Y$ to $Z$, and $A\subseteq X$.

    Further suppose $x\in\image{S\circ R}{A}$, by definition of image there exists $c\in A$ such that $(c, x)\in
    S\circ R$.
    By definition of composition there exists some $y\in Y$ such that $(c, y)\in R$ and $(y, x)\in S$.
    Then, by definition of image $y\in\image{R}{A}$, so by definition of image $x\in\image{S}{\image{R}{A}}$.
    Therefore $\image{S\circ R}{A}\subseteq\image{S}{\image{R}{A}}$.

    Further suppose $x\in\image{S}{\image{R}{A}}$, by definition of image there is some $y\in Y$ such
    that $y\in\image{R}{A}$, and by definition of image $x\in\image{S}{Y}$.
    By definition of composition there exists some $y\in Y$ such that $(c, y)\in R$ and $(y, x)\in S$.
    Therefore $\image{S}{\image{R}{A}}\subseteq\image{S\circ R}{A}$.

    Thus, by definition of equality $\image{S\circ R}{A}=\image{S}{\image{R}{A}}$.


    \medskip
    \noindent
    \textbf{5.3.11}\\
    False:\\
    $A=\{a_1, a_2\}$, $B=\{b_1\}$, $R=\{(a_1, b_2)\}$

    \medskip
    \noindent
    \textbf{5.4.3}\\
    \textbf{Proof.} Suppose $x\in\mathbb{R}$, by definition of less than or equal to $x\leq x$.

    Hence, by definition of reflexive, $\leq$ is reflexive.

    \medskip
    \noindent
    \textbf{5.4.4}\\
    $0\leq 1$ but $1\nleq 0$

    \medskip
    \noindent
    \textbf{5.4.5}\\
    \textbf{Proof.} Suppose $a, b, c\in\mathbb{R}$, and suppose that $a\leq b$ and $b\leq c$.
    By definition of less than or equal to $b\leq b$, therefore by definition of less than or equal $a\leq c$.

    Hence $\leq$ is transitive.

    \medskip
    \noindent
    \textbf{5.4.22}\\
    \textbf{Proof.} Suppose $(x, y)\in R\cup R^{-1}$.
    By definition of union $(x, y)\in R$ or $(x, y)\in R^{-1}$.

    Suppose $(x, y)\in R$, by definition of inverse $(y, x)\in R^{-1}$.

    Suppose $(x, y)\in R^{-1}$, by definition of inverse $(y, x)\in(R^{-1})^{-1}$, by definition of inverse of
    inverse $(y, x)\in R$.

    Therefore, by division into cases and by definition of symmetric $R\cup R^{-1}$ is symmetric.

    \medskip
    \noindent
    \textbf{5.4.24}\\
    \textbf{Proof.} Suppose $(a, b),(b,c)\in R\cap S$.
    By definition of intersect $(a, b),(b,c)\in R$ and $(a, b),(b, c)\in S$.
    By definition of transitive $(a, c)\in R$ and by definition of transitive $(a, c)\in S$.
    Therefore, by definition of intersect $(a, c)\in R\cap S$, thus $R\cap S$ is transitive.

    \medskip
    \noindent
    \textbf{5.4.25}\\
    Suppose $R$ is reflexive.
    Further suppose $a\in A$.
    By definition of reflex ive $(a, a)\in R$, so by definition of image $a\in\image{R}{A}$.

    Therefore by definition of subset $A\subseteq\image{R}{A}$.

    \medskip
    \noindent
    \textbf{5.5.7}\\
    \textbf{Proof.} Suppose $R$ and $S$ are equivalence relations, also suppose $(x,y)\in R\cap S$.
    By definition of intersect $(x,y)\in R$ and $(x, y)\in S$.

    By definition of reflexivity and definition of equivalence $(x,x)\in R$ and $(x, x)\in S$.
    Therefore by definition of reflexivity and definition of intersect $R\cap S$ is reflexive.

    By definition of symmetry and definition of equivalence $(y,x)\in R$ and $(y,x)\in S$.
    Therefore by definition of symmetry and definition of intersect $R\cap S$ is symmetric.

    Further suppose $(y, z)\in R\cap S$.
    By definition of transitivity and definition of equivalence $(x, z)\in R$ and $(x, z)\in S$.
    Therefore by definition of transitivity and definition of intersect $R\cap S$ is transitive.

    Hence, by definition of equivalence $R\cap S$ is an equivalence relation.

    \medskip
    \noindent
    \textbf{5.5.9}\\
    $A=\{1, 2, 3\}$
    $R=\{(1, 1), (1, 2), (2, 1), (2, 2)\}$

    \medskip
    \noindent
    \textbf{5.5.10}\\
    \textbf{Proof.} Suppose $R$ is symmetric and transitive, and suppose the domain of $A$ is equal to the domain of
    $\image{R}{A}$, further suppose $x, y\in A$ such that $(x, y)\in R$.

    By definition of symmetric $(y,x)\in R$, so by definition of transitive $(x,x)\in R$.

    Therefore, by definition of reflexive $R$ is reflexive.
\end{document}