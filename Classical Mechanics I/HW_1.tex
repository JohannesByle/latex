%! Author = Johannes Byle
%! Date = 8/30/2021

% Preamble
\documentclass[12pt]{article}
\title{Classical Mechanics Assignment \#1}
\author{Johannes Byle}

% Packages
\usepackage{amsmath}
\usepackage[margin=0.75in]{geometry}
\usepackage{lipsum}
\usepackage{braket}

% Document
\begin{document}
    \maketitle
    \begin{enumerate}
        \item
        \begin{enumerate}
            \item
            Starting with the forces, simplifying, and substituting variables:
            \begin{gather*}
                F_{\text{total}}=F_{\text{spring}}+F_{\text{damping}}+F_{\text{piston}}\\
                m\ddot{x}=-kx-m\nu\dot{x}+kX(t)\\
                \ddot{x}+\frac{k}{m}x+\nu\dot{x}=\frac{k}{m}(t)\\
                \ddot{x}+\nu\dot{x}+\omega_0^2 x=F_0(t)
            \end{gather*}
            \item
            Complementary solution:
            \begin{gather*}
                x(t)=e^{-\beta t}\left[A_1 e^{-i\sqrt{\omega_0^2-\beta^2}t}+A_2 e^{i\sqrt{\omega_0^2-\beta^2}t}\right]
            \end{gather*}
            Particular solution:
            \begin{gather*}
                X(t)=X_0 e^{\alpha t}\cos(\omega t)\\
                \dot{X}(t)=e^{\alpha t}X_0\left(\alpha\cos(\omega t)-\omega\sin(\omega t)\right)\\
                \ddot{X}(t)=X_0 e^{\alpha t} \left(\left(\alpha^2-\omega_0^2\right) \cos (t \omega)-2 \alpha \omega \sin (t \omega)\right)\\
            \end{gather*}
            Particular solution continued: substituting back into equation:
            \begin{gather*}
                X_0 e^{\alpha t} \left(\left(\alpha^2-\omega_0^2\right) \cos (t \omega)-2 \alpha \omega \sin (t \omega)\right)+\\
                \nu e^{\alpha t}X_0\left(\alpha\cos(\omega t)-\omega\sin(\omega t)\right)+\\
                \omega_0^2 X_0 e^{\alpha t}\cos(\omega t)=\omega^2 e^{\alpha t}\cos(\omega t)\\
            \end{gather*}
            Particular solution continued: finding $X_0$:
            \begin{gather*}
                \omega_0^2 X_0 e^{\alpha t}\cos(\omega t)=\omega^2 e^{\alpha t}\cos(\omega t)\\
            \end{gather*}
        \end{enumerate}
        \item
        \begin{enumerate}
            \item
            Assuming the "ripples" can be modeled by a sinusoidal function, the motion of the car can be described by the driven damped oscillator discussed in class.
            In this case the equation for the resonance frequency:
            \begin{gather*}
                \omega_r=\sqrt{\omega_0^2+2\beta^2}\\
                \frac{v}{x_\text{spacing}}=\sqrt{\omega_0^2+2\beta^2}\\
                v=x_\text{spacing}\sqrt{\omega_0^2+2\beta^2}\\
            \end{gather*}
            Since $\omega_0=\sqrt{\frac{k}{m}}$ and $\beta=\frac{b}{2m}$:
            \begin{gather*}
                \omega_r=\sqrt{\omega_0^2+2\beta^2}\\
                \frac{v}{x_\text{spacing}}=\sqrt{\omega_0^2+2\beta^2}\\
                v=x_\text{spacing}\sqrt{\frac{k}{m}+\frac{b^2}{4m^2}}\\
            \end{gather*}
            Plugging in some reasonable values we can check whether or not our equations make physical sense.
            I used the mass of my car $m=1565$ kg, and damping $b=2000$ N s/m and stiffness $k=22,000$ N/m parameters from a research paper.\footnote{https://journals.sagepub.com/doi/full/10.1177/1687814016648638 Analysis of suspension with variable stiffness and variable damping force for automotive applications}
            \begin{gather*}
                v=2\sqrt{\frac{22000}{1565}+\frac{2000^2}{4\cdot1565^2}}\approx7.6\text{ m/s}\approx17\text{ mph}\\
            \end{gather*}
            This is a reasonable number and the units make sense:
            \begin{gather*}
                v=m\sqrt{\frac{N}{m\cdot kg}+\frac{N^2\cdot s^2}{m^2\cdot kg^2}}\\
                v=m\sqrt{\frac{kg\cdot m}{m\cdot kg\cdot s^2}+\frac{kg^2\cdot m^2\cdot s^2}{m^2\cdot kg^2\cdot s^4}}\\
                v=m\sqrt{\frac{1}{s^2}+\frac{1}{s^2}}=\frac{m}{s}\\
            \end{gather*}
            \item
            So that the suspension doesn't fall apart I probably want a damping constant that only allows resonances at very high speeds, say above $100$ mph:
            \begin{gather*}
                \sqrt{4m^2\left(\frac{v^2}{x^2_\text{spacing}}-\frac{k}{m}\right)}=b\\
                \sqrt{4\cdot1565^2\left(\frac{45}{2}-\frac{22000}{1565}\right)}\approx 9100\text{ N s/m}\\
            \end{gather*}
        \end{enumerate}
    \end{enumerate}

\end{document}