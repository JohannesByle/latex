%! Author = johannes
%! Date = 9/22/21

% Preamble
\documentclass[12pt]{article}
\title{Classical Assignment \#5}
\author{Johannes Byle}
\newcommand{\p}[2]{\frac{\partial #1}{\partial #2}}
\newcommand{\der}[2]{\frac{d #1}{d #2}}
\newcommand{\Lag}[3]{\p{L}{#1}-\der{}{t}\p{L}{\dot{#1}}=0\\ \p{L}{#1}=#2,\quad \der{}{t}\p{L}{\dot{#1}}=#3\\ #2-#3=0}
\newcommand{\Lagq}[4]{{\p{L}{#1}-\der{}{t}\p{L}{\dot{#1}}+\lambda\p{f}{r}=0\\ \p{L}{#1}=#2,\quad \der{}{t}\p{L}{\dot{#1}}=#3,\quad \lambda\p{f}{r}=#4\\ #2-#3+\lambda#4=0}}
% Packages
\usepackage{amsmath}
\usepackage[margin=0.75in]{geometry}
\usepackage{lipsum}
\usepackage{physics}

% Document
\begin{document}
    \maketitle
    \begin{enumerate}
        \item
        \begin{enumerate}
            \item Using spherical coordinates, and setting our coordinates such that the mass can be described by only $\theta$ and $r$ we get the following equations for the kinetic and potential energies and the constraint $f$:
            \begin{gather*}
                T=\frac{m}{2}\left(\dot{r}^2+r^2\dot{\theta}^2\right)\quad U=mgr\cos\theta\\
                L=\frac{m}{2}\left(\dot{r}^2+r^2\dot{\theta}^2\right)-mgr\cos\theta\\
                f=r-R=0
            \end{gather*}
            Solving the Lagrangian for $r$:
            \begin{gather*}
                \Lagq{r}{m r\dot{\theta}^2-g m \cos\theta}{m\ddot{r}}{1}
            \end{gather*}
            Solving the Lagrangian for $\theta$:
            \begin{gather*}
                \Lag{\theta}{gmr\sin\theta}{2mr\dot{r}\dot{\theta}+mr^2\ddot{\theta}}
            \end{gather*}
            Since we know that $r=R$ and does not change, i.e. $\dot{r}=\ddot{r}=0$, the equations simplify to:
            \begin{gather*}
                mR\dot{\theta}^2-mg\cos\theta+\lambda=0\\
                mgR\sin\theta-mR^2\ddot{\theta}=0
            \end{gather*}
            Solving for $\ddot{\theta}$ and integrating we can use substitution to solve for $\lambda$:
            \begin{gather*}
                \ddot{\theta}=\frac{g\sin\theta}{R}\\
                \int \ddot{\theta}d\dot{\theta}=\frac{g}{R}\int\sin\theta d\theta\quad \text{from Marion page 253}\\
                \frac{\dot{\theta}^2}{2}=\frac{-g\cos\theta}{R}+\frac{g}{R}
            \end{gather*}
            \begin{align*}
                \lambda&=mg\cos\theta-2mR\left( \frac{-g\cos\theta}{R}+\frac{g}{R} \right)\\
                &=mg\cos\theta+2mg\cos\theta-2mg\\
                \lambda &=mg(3\cos\theta-2)
            \end{align*}
            \item We can find the height at which the ball will fall off the sphere using the fact that once the force of constraint goes to zero the ball will fall off the sphere, so solving for where $\lambda=0$:
            \begin{align*}
                0&=mg(3\cos\theta-2)\\
                0&=3\cos\theta-2\\
                \theta_f&=\cos^{-1}\left(\frac{2}{3}\right)
            \end{align*}
            The height is now trivial to find from the angle: $h=R+R\sin\theta_f$
            \item Repeating the same process as above, but adding an extra constraint for the rolling without slipping condition, updating the radius, and adding rotational kinetic energy term $\frac{1}{2}\frac{2}{5}a^2\frac{\dot{r}^2+r^2\dot{\theta}^2}{a^2}$:
            \begin{gather*}
                T=\frac{m}{2}\frac{7}{5}\left(\dot{r}^2+r^2\dot{\theta}^2\right)\quad U=mgr\cos\theta\\
                L=\frac{m}{2}\frac{7}{5}\left(\dot{r}^2+r^2\dot{\theta}^2\right)-mgr\cos\theta\\
                f=r-(R+a)=0
            \end{gather*}
            Since this is only a change of a constant, which we can simply pull out of all the derivatives, we can simply add it to the final solution from the previous problem, along with the updated value of $R$:
            \begin{gather*}
                \frac{7}{5}m(R+a)\dot{\theta}^2-mg\cos\theta+\lambda=0\\
                mg(R+a)\sin\theta-\frac{7}{5}m(R+a)^2\ddot{\theta}=0
            \end{gather*}
            This will also change our value of $\dot{\theta}^2$:
            \begin{gather*}
                \frac{\dot{\theta}^2}{2}=\frac{5}{7}\frac{-g\cos\theta}{R+a}+\frac{5}{7}\frac{g}{R+a}
            \end{gather*}
            Which changes our value for $\lambda$ in the same simple way:
            \begin{align*}
                \lambda&=mg\cos\theta-2m(R+a)\left( \frac{5}{7}\frac{-g\cos\theta}{R+a}+\frac{5}{7}\frac{g}{R+a} \right)\\
                &=mg\cos\theta+\frac{5}{7}2mg\cos\theta-\frac{5}{7}2mg\\
                \lambda &=\frac{mg}{7}(17\cos\theta-10)
            \end{align*}
            This means that our angle changes simply to $\theta_f&=\cos^{-1}\left(\frac{10}{17}\right)$ and the equation for height also changes slightly: $h=R+(R+a)\sin\theta_f$.

        \end{enumerate}
    \end{enumerate}

\end{document}