%! Author = Johannes Byle
%! Date = 9/23/2021

% Preamble
\documentclass[12pt]{article}
\title{}
\author{Johannes Byle}

% Packages
\usepackage{amsmath}
\usepackage[margin=0.75in]{geometry}
\usepackage{lipsum}
\usepackage{physics}
\newcommand{\p}[2]{\frac{\partial #1}{\partial #2}}
\newcommand{\der}[2]{\frac{d #1}{d #2}}

% Document
\begin{document}
    \maketitle
    \begin{enumerate}
        \item
        \begin{enumerate}
            \item Using conservation of energy we can find the velocity:
            \begin{gather*}
                \frac{1}{2}mv^2=mgy\\
                v=\sqrt{2gy}
            \end{gather*}
            This gives the following time of descent:
            \begin{gather*}
                t=\int \frac{ds}{v}=\frac{1}{2g}\int \frac{\sqrt{dx^2+dy^2}}{\sqrt{y}}
            \end{gather*}
            Converting to an integral of $dy$:
            \begin{gather*}
                t=\frac{1}{2g}\int dy\frac{\sqrt{\left(\dv{x}{y}\right)^2+1}}{\sqrt{y}}=\int_{y_1}^{y_2}dy\sqrt{\frac{x'^2+1}{y}}
            \end{gather*}
            \item Taking the lagrangian:
            \begin{gather*}
                \p{F}{x}-\der{}{y}\left( \p{F}{\dot{x}} \right)\\
                \p{F}{x}=0\\
                \der{}{y}\left( \p{F}{\dot{x}} \right)=\der{}{y}\left(\frac{1}{\sqrt{2g}}\frac{x'}{y}\sqrt{\frac{y}{x'^2+1}}\right)
            \end{gather*}
            Since $\p{F}{x}=0$ is zero we can simply integrate:
            \begin{gather*}
                \frac{1}{\sqrt{2g}}\frac{x'}{y}\sqrt{\frac{y}{x'^2+1}}=c
            \end{gather*}
            Solving for $x'$:
            \begin{gather*}
                \der{x}{y}=\frac{c\sqrt{2gy}}{\sqrt{1-2c^2 gy}}
            \end{gather*}
            Integrating:
            \begin{gather*}
                y_1-y_2=\int \frac{\sqrt{1-2c^2 gy}}{c\sqrt{2gy}}dx
            \end{gather*}
            Im pretty sure this integral can be solved with the change of variables $x=\frac{c^2}{4g}(\theta-\sin\theta)$ and $y=\frac{c^2}{4g}\left( 1-\cos\theta \right)$.
        \end{enumerate}
        \item Since this is the Lagrangian, writing it in the form of the action $I$ and differentiating $I(\epsilon)$ with respect to $\epsilon$:
        \[
            \frac{dI}{d\epsilon}=\int_{t_A}^{t_B}\left[\frac{\delta f}{\delta q}\frac{dq}{d\epsilon}+\frac{\delta f}{\delta q'}\frac{dq'}{d\epsilon}+\frac{\delta f}{\delta q''}\frac{dq''}{d\epsilon}\right]
        \]
        Since $q'$ endpoints are prescribed, the integration by parts trick used when deriving the Lagrangian will work on both the $\frac{\delta f}{\delta y'}\frac{dy'}{d\epsilon}$ and $\frac{\delta f}{\delta y''}\frac{dy''}{d\epsilon}$ terms.
        \[
            \frac{dI}{d\epsilon}=\int_{t_A}^{t_B}\left[\frac{\delta f}{\delta q}-\frac{d}{dt}\left(\frac{\delta f}{\delta q'}\right)-\frac{d^2}{dt^2}\left(\frac{\delta f}{\delta q'}\right)\right]\frac{dq}{d\epsilon}dx
        \]
        This requires that $y(x,\epsilon)$ and all its derivatives through third order are continuous functions of $x$ and $\epsilon$
        \item
        \begin{enumerate}
            \item The particle is confined to the hoop and thus can only move around the hoop and thus is:
            \begin{gather*}
                T=\frac{1}{2}m\left( R^2\sin^2\theta\omega^2+R\dot{\theta}^2 \right)\\
                V=-mgy=-mgR\cos\theta\\
                L=T-V=\frac{1}{2}m\left( R^2\sin^2\theta\omega^2+R\dot{\theta}^2 \right)-mgR\cos\theta
            \end{gather*}
            Solving the Lagrangian:
            \begin{gather*}
                \p{L}{\theta}-\der{}{t}\left( \p{L}{\dot{\theta}} \right)=0\\
                \p{L}{\theta}=mR^2 \omega^2\sin\theta\cos\theta+mgR\sin\theta\\
                \der{}{t}\left( \p{L}{\dot{\theta}} \right)=mR\ddot{\theta}\\
                mR^2 \omega^2\sin\theta\cos\theta+mgR\sin\theta-mR\ddot{\theta}=0\\
                R\omega^2\sin\theta\cos\theta+g\sin\theta-\ddot{\theta}=0
            \end{gather*}
            \item The bead is stationary when $\ddot{\theta}=0$ which means:
            \begin{gather*}
                R\omega^2\sin\theta\cos\theta=g\sin\theta
            \end{gather*}
            This means that either $\sin\theta=0$ or $\cos\theta=\frac{g}{R\omega^2}$.\\
            Solving for $\omega$:
            \begin{gather*}
                \omega=\sqrt{\frac{\ddot{\theta}-g\sin\theta}{R\sin\theta\cos\theta}}
            \end{gather*}
        \end{enumerate}

    \end{enumerate}

\end{document}