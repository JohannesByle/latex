\documentclass[12pt]{article}
\usepackage[hscale=0.7,vscale=0.8]{geometry}
\usepackage{amsmath}
\usepackage{pgfplots}
\usepackage{tikz}
\title{Notes 8/23}
\author{Johannes Byle}
\begin{document}
    \maketitle

    \subsection*{Simple Harmonic Oscillator}
    \begin{enumerate}
        \item Mass on a spring (Hooke's law)
        \item Pendulum's with small angle approximation
        \item Quantum mechanics
    \end{enumerate}
    \[
        SHO: \ddot{x}=\omega_{0}^{2}x
    \]

    \subsection*{Force has some associated potential}
    \[
        F=-\frac{\delta V}{\delta x}\rightarrow V(x)=\frac{1}{2}m\omega_{0}^{2}x^{2}
    \]

    \subsection*{Solution}
    \[
        A_0\cos(\omega_0t-\delta)
    \]

    \subsection*{Phase Space}
    \[
        (x,p)\rightarrow \text{position momentum space}
    \]
    \begin{center}
        \begin{tikzpicture}
            \begin{axis}[samples=100, xmin=-1.5, xmax=1.5, xlabel=x, ylabel=p]
                \addplot[color=black, domain=-2:2]{sqrt(1-x^2)};
                \addplot[color=black, domain=-2:2]{-sqrt(1-x^2)};
            \end{axis}
        \end{tikzpicture}
    \end{center}

    \subsection*{Phase Portrait}
    A collection of trajectories & orbits in phase space, only really possible with orbits

    \subsection*{Damping}
    \[
        m\ddot{x}+b\dot{x}+kx=0
    \]
    Linear damping is called viscous damping.
\end{document}