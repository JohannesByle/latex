%! Author = johannes
%! Date = 10/25/21

% Preamble
\documentclass[10pt]{article}
\title{Classical Assignment \#7}
\author{Johannes Byle}

% Packages
\usepackage{amsmath}
\usepackage[margin=0.75in]{geometry}
\usepackage{lipsum}
\usepackage{ifthen}
\usepackage{tikz}
\usepackage{pgfplots}
\usepackage{graphicx}
\usepackage{csquotes}
\usetikzlibrary{math}
\usetikzlibrary{angles, quotes}
\usepackage{amssymb, graphics, setspace}

% Commands
\newcommand{\p}[2]{\frac{\partial #1}{\partial #2}}
\newcommand{\der}[2]{\frac{d #1}{d #2}}
\newcommand{\Lag}[3]{
    \p{L}{#1}-\der{}{t}\p{L}{\dot{#1}}=0\\
    \p{L}{#1}=#2,\quad \der{}{t}\p{L}{\dot{#1}}=#3\\
    #2-#3=0
}
\newcommand{\Lagq}[4]{
    \p{L}{#1}-\der{}{t}\p{L}{\dot{#1}}+\lambda\p{f}{r}=0\\
    \p{L}{#1}=#2,\quad \der{}{t}\p{L}{\dot{#1}}=#3,\quad \p{f}{#1}=#4\\
    #2-#3\ifthenelse{#4=1}{+\lambda}{\ifthenelse{#4=0}{+0}{+\lambda#4}}=0
}
\newcommand{\op}[1]{\pmb{\text{#1}}}

% Document
\begin{document}
    \maketitle
    \begin{enumerate}
        \item From Goldstein page 153 we know that we can go from body coordinates to space axes using the following relation $\op{x}=\op{A}^{-1}\op{x}'$.
        Since $\op{A}=\op{BCD}$:
        \begin{gather*}
            \op{BCD}=
            \left(
            \begin{array}{ccc}
                \cos (\phi )  & \sin (\phi ) & 0 \\
                -\sin (\phi ) & \cos (\phi ) & 0 \\
                0             & 0            & 1 \\
            \end{array}
            \right)
            \left(
            \begin{array}{ccc}
                1 & 0               & 0              \\
                0 & \cos (\theta )  & \sin (\theta ) \\
                0 & -\sin (\theta ) & \cos (\theta ) \\
            \end{array}
            \right)
            \left(
            \begin{array}{ccc}
                \cos (\psi )  & \sin (\psi ) & 0 \\
                -\sin (\psi ) & \cos (\psi ) & 0 \\
                0             & 0            & 1 \\
            \end{array}
            \right)\\
            \op{A}=
            \left(
            \begin{array}{ccc}
                \cos (\psi ) \cos (\phi )-\cos (\theta ) \sin (\psi ) \sin (\phi )  & \cos (\theta ) \sin (\psi ) \cos (\phi )+\cos (\psi ) \sin (\phi ) & \sin (\theta ) \sin (\psi ) \\
                -\cos (\theta ) \cos (\psi ) \sin (\phi )-\sin (\psi ) \cos (\phi ) & \cos (\theta ) \cos (\psi ) \cos (\phi )-\sin (\psi ) \sin (\phi ) & \sin (\theta ) \cos (\psi ) \\
                \sin (\theta ) \sin (\phi )                                         & \sin (\theta ) (-\cos (\phi ))                                     & \cos (\theta )              \\
            \end{array}
            \right)\\
            \op{A}^{-1}=
            \left(
            \begin{array}{ccc}
                \cos (\psi ) \cos (\phi )-\cos (\theta ) \sin (\psi ) \sin (\phi ) & -\cos (\theta ) \cos (\psi ) \sin (\phi )-\sin (\psi ) \cos (\phi ) & \sin (\theta ) \sin (\phi ) \\
                \cos (\theta ) \sin (\psi ) \cos (\phi )+\cos (\psi ) \sin (\phi ) & \cos (\theta ) \cos (\psi ) \cos (\phi )-\sin (\psi ) \sin (\phi ) & \sin (\theta ) (-\cos (\phi )) \\
                \sin (\theta ) \sin (\psi )                                        & \sin (\theta ) \cos (\psi )                                         & \cos (\theta )                 \\
            \end{array}
            \right)\\
            \omega_{bf}=
            \begin{pmatrix}
                \theta ' \cos (\psi )+\sin (\theta ) \sin (\psi ) \phi ' \\
                \sin (\theta ) \cos (\psi ) \phi '-\theta ' \sin (\psi ) \\
                \cos (\theta ) \phi '+\psi '
            \end{pmatrix}\\
            \op{A}^{-1}\omega_{bf}=
            \begin{pmatrix}
                \theta ' \cos (\phi )+\sin (\theta ) \psi ' \sin (\phi ) \\
                \theta ' \sin (\phi )-\sin (\theta ) \psi ' \cos (\phi ) \\
                \cos (\theta ) \psi '+\phi '
            \end{pmatrix}
        \end{gather*}
        \item
        \begin{enumerate}
            \item Simply plugging in equation (5.22) from Goldstein into Mathematica:\\
            \linebreak
            \parbox{\textwidth}{
                \text{Clear}[\text{Global$\grave{ }$*}]\\
                x = \{r \text{Cos}[\theta ],r \text{Sin}[\theta ], z\};\\
                f[\text{j$\_$},\text{k$\_$}]\text{:=}(x[[1]]{}^{\wedge}2+x[[2]]{}^{\wedge}2+z{}^{\wedge}2) \text{KroneckerDelta}[j,k]-x[[j]] x[[k]];\\
                i[\text{j$\_$},\text{k$\_$},\text{hi$\_$},\text{hf$\_$}]\text{:=}M \text{Integrate}[\text{Integrate}[\text{Integrate}[f[j,k] r,\{z,((\text{hf}-\text{hi})/R)
                r+\text{hi},\text{hf}\}],\{r,0,R\}],\{\theta ,0,2 \text{Pi}\}]\\
                \text{ITensor}[\text{hi$\_$},\text{hf$\_$}]\text{:=}\text{Table}[\text{FullSimplify}[i[j,k,\text{hi},\text{hf}]],\{j,1,3\}, \{k,1,3\}];\\
                \text{com} = \text{ITensor}[-(3/4)h,(1/4)h];\\
                \text{origin} = \text{ITensor}[0,h];\\
                \text{MatrixForm}[\text{com}]\\
                \text{MatrixForm}[\text{origin}]\\
                \text{RVec} = \{0,0,-3/4\};\\
                \text{steiner}[\text{j$\_$},\text{k$\_$}]\text{:=}(\text{origin}[[j,k]] -M (\text{Norm}[\text{RVec}]{}^{\wedge}2 \text{KroneckerDelta}[j,k] - \text{RVec}[[j]]
                \text{RVec}[[k]]))\\
                \text{MatrixForm}[\text{Table}[\text{TrueQ}[\text{com}[[j,k]] \text{==} \text{steiner}[j,k]],\{j,1,3\},\{k,1,3\}]]
            }
            \begin{enumerate}
                \item\begin{doublespace}
                         \noindent\(\left(
                         \begin{array}{ccc}
                             \frac{1}{80} h M \pi  R^2 \left(h^2+4 R^2\right) & 0                                              & 0                       \\
                             0                                              & \frac{1}{80} h M \pi  R^2 \left(h^2+4 R^2\right) & 0                       \\
                             0                                              & 0                                              & \frac{1}{10} h M \pi  R^4 \\
                         \end{array}
                         \right)\)
                \end{doublespace}
                \item\begin{doublespace}
                         \noindent\(\left(
                         \begin{array}{ccc}
                             \frac{1}{20} h M \pi  R^2 \left(4 h^2+R^2\right) & 0                                              & 0                       \\
                             0                                              & \frac{1}{20} h M \pi  R^2 \left(4 h^2+R^2\right) & 0                       \\
                             0                                              & 0                                              & \frac{1}{10} h M \pi  R^4 \\
                         \end{array}
                         \right)\)
                \end{doublespace}
            \end{enumerate}
            \item This array should be all True, but I think I've made a mistake in my definition of the Steiner equation.
            \begin{doublespace}
                \noindent\(\left(
                \begin{array}{ccc}
                    \text{False} & \text{True}  & \text{True} \\
                    \text{True}  & \text{False} & \text{True} \\
                    \text{True}  & \text{True}  & \text{True} \\
                \end{array}
                \right)\)
            \end{doublespace}
        \end{enumerate}
        \item
        \item The kinetic energy will have a term from the translation, which will be a rotation around the apex of the cone.
        Defining $\theta$ as the angle between the cone and the x-axis (the cone is resting on the x-y plane) we get the following translational kinetic energy:
        \begin{gather*}
            V=\dot{\theta}\frac{3}{4}h\cos\alpha\\
            T=\frac{9}{32}M\dot{\theta}^2 h^2\cos^2\alpha
        \end{gather*}
        For the rotational kinetic energy we need to find the angular velocity which we can define from $\dot{\theta}$:
        \begin{gather*}
            \omega=\frac{4V}{3h\sin\alpha}=\frac{\dot{\theta}\cos\alpha}{\sin\alpha}=\dot{\theta}\cot\alpha
        \end{gather*}
        Using the moment of inertia tensor derived in the previous problem:
        \begin{gather*}
            I=
            \begin{bmatrix}
                \frac{1}{20} h M \pi  R^2 \left(4 h^2+R^2\right) & 0                                              & 0                       \\
                0                                              & \frac{1}{20} h M \pi  R^2 \left(4 h^2+R^2\right) & 0                       \\
                0                                              & 0                                              & \frac{1}{10} h M \pi  R^4 \\
            \end{bmatrix}
        \end{gather*}
        Substituting $h\sin\alpha=R$:
        \begin{gather*}
            I=
            \begin{bmatrix}
                \frac{1}{20} h M \pi  R^2 \left(4 h^2+h\sin^2\alpha\right) & 0                                                                  & 0                                 \\
                0                                                        & \frac{1}{20} h M \pi  h\sin^2\alpha \left(4 h^2+h\sin^2\alpha\right) & 0                                 \\
                0                                                        & 0                                                                  & \frac{1}{10} h M \pi  h\sin^4\alpha \\
            \end{bmatrix}
        \end{gather*}
        Finding the rotational kinetic energy with $T=\omega\cdot I\cdot\omega$:
        \begin{gather*}
            \omega\cdot I\cdot\omega=
            \begin{bmatrix}
                \dot{\theta} \\
                0            \\
                \dot{\theta}\cot\alpha
            \end{bmatrix}
            \begin{bmatrix}
                \frac{1}{20} h M \pi  R^2 \left(4 h^2+h\sin^2\alpha\right) & 0                                                                  & 0                                 \\
                0                                                        & \frac{1}{20} h M \pi  h\sin^2\alpha \left(4 h^2+h\sin^2\alpha\right) & 0                                 \\
                0                                                        & 0                                                                  & \frac{1}{10} h M \pi  h\sin^4\alpha \\
            \end{bmatrix}
            \begin{bmatrix}
                \dot{\theta} \\
                0            \\
                \dot{\theta}\cot\alpha
            \end{bmatrix}\\
            \omega\cdot I\cdot\omega=\dot{\theta}^2 \frac{1}{20} h M \pi  R^2 \left(4 h^2+h\sin^2\alpha\right)+\dot{\theta}^2\cot^2\alpha\frac{1}{10} h M \pi  h\sin^4\alpha
        \end{gather*}
        The final kinetic energy is then:
        \begin{gather*}
            T=\frac{9}{32}M\dot{\theta}^2 h^2\cos^2\alpha+\frac{1}{2}\left[\dot{\theta}^2 \frac{1}{20} h M \pi  R^2 \left(4 h^2+h\sin^2\alpha\right)+\dot{\theta}^2\cot^2\alpha\frac{1}{10} h M \pi  h\sin^4\alpha\right]
        \end{gather*}
    \end{enumerate}
\end{document}