%! Author = johannes
%! Date = 10/25/21

% Preamble
\documentclass[11pt]{article}
\title{Classical Assignment \#7}
\author{Johannes Byle}

% Packages
\usepackage{amsmath}
\usepackage[margin=0.75in]{geometry}
\usepackage{lipsum}
\usepackage{ifthen}
\usepackage{tikz}
\usepackage{pgfplots}
\usepackage{graphicx}
\usepackage{csquotes}
\usetikzlibrary{math}
\usetikzlibrary{angles, quotes}

% Commands
\newcommand{\p}[2]{\frac{\partial #1}{\partial #2}}
\newcommand{\der}[2]{\frac{d #1}{d #2}}
\newcommand{\Lag}[3]{
  \p{L}{#1}-\der{}{t}\p{L}{\dot{#1}}=0\\
  \p{L}{#1}=#2,\quad \der{}{t}\p{L}{\dot{#1}}=#3\\
  #2-#3=0
}
\newcommand{\Lagq}[4]{
  \p{L}{#1}-\der{}{t}\p{L}{\dot{#1}}+\lambda\p{f}{r}=0\\
  \p{L}{#1}=#2,\quad \der{}{t}\p{L}{\dot{#1}}=#3,\quad \p{f}{#1}=#4\\
  #2-#3\ifthenelse{#4=1}{+\lambda}{\ifthenelse{#4=0}{+0}{+\lambda#4}}=0
}
\newcommand{\op}[1]{\pmb{\text{#1}}}

% Document
\begin{document}
  \maketitle
  \begin{enumerate}
    \item From Goldstein page 153 we know that we can go from body coordinates to space axes using the following relation $\op{x}=\op{A}^{-1}\op{x}'$.
    Since $\op{A}=\op{BCD}$:
    \begin{gather*}
      \op{BCD}=
      \left(
      \begin{array}{ccc}
        \cos (\phi )  & \sin (\phi ) & 0 \\
        -\sin (\phi ) & \cos (\phi ) & 0 \\
        0             & 0            & 1 \\
      \end{array}
      \right)
      \left(
      \begin{array}{ccc}
        1 & 0               & 0              \\
        0 & \cos (\theta )  & \sin (\theta ) \\
        0 & -\sin (\theta ) & \cos (\theta ) \\
      \end{array}
      \right)
      \left(
      \begin{array}{ccc}
        \cos (\psi )  & \sin (\psi ) & 0 \\
        -\sin (\psi ) & \cos (\psi ) & 0 \\
        0             & 0            & 1 \\
      \end{array}
      \right)\\
      \op{A}=
      \left(
      \begin{array}{ccc}
        \cos (\psi ) \cos (\phi )-\cos (\theta ) \sin (\psi ) \sin (\phi )  & \cos (\theta ) \sin (\psi ) \cos (\phi )+\cos (\psi ) \sin (\phi ) & \sin (\theta ) \sin (\psi ) \\
        -\cos (\theta ) \cos (\psi ) \sin (\phi )-\sin (\psi ) \cos (\phi ) & \cos (\theta ) \cos (\psi ) \cos (\phi )-\sin (\psi ) \sin (\phi ) & \sin (\theta ) \cos (\psi ) \\
        \sin (\theta ) \sin (\phi )                                         & \sin (\theta ) (-\cos (\phi ))                                     & \cos (\theta )              \\
      \end{array}
      \right)\\
      \op{A}^{-1}=
      \left(
      \begin{array}{ccc}
        \cos (\psi ) \cos (\phi )-\cos (\theta ) \sin (\psi ) \sin (\phi ) & -\cos (\theta ) \cos (\psi ) \sin (\phi )-\sin (\psi ) \cos (\phi ) & \sin (\theta ) \sin (\phi ) \\
        \cos (\theta ) \sin (\psi ) \cos (\phi )+\cos (\psi ) \sin (\phi ) & \cos (\theta ) \cos (\psi ) \cos (\phi )-\sin (\psi ) \sin (\phi ) & \sin (\theta ) (-\cos (\phi )) \\
        \sin (\theta ) \sin (\psi )                                        & \sin (\theta ) \cos (\psi )                                         & \cos (\theta )                 \\
      \end{array}
      \right)\\
      \omega_{bf}=
      \begin{pmatrix}
        \theta ' \cos (\psi )+\sin (\theta ) \sin (\psi ) \phi ' \\
        \sin (\theta ) \cos (\psi ) \phi '-\theta ' \sin (\psi ) \\
        \cos (\theta ) \phi '+\psi '
      \end{pmatrix}\\
      \op{A}^{-1}\omega_{bf}=
      \begin{pmatrix}
        \theta ' \cos (\phi )+\sin (\theta ) \psi ' \sin (\phi ) \\
        \theta ' \sin (\phi )-\sin (\theta ) \psi ' \cos (\phi ) \\
        \cos (\theta ) \psi '+\phi '
      \end{pmatrix}
    \end{gather*}
    \item
  \end{enumerate}

\end{document}