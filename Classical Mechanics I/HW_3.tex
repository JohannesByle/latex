%! Author = johannes
%! Date = 9/14/21

% Preamble
\documentclass[12pt]{article}
\title{Classical Mechanics Assignment \#3}
\author{Johannes Byle}
\newcommand{\p}[2]{\frac{\partial #1}{\partial #2}}
\newcommand{\der}[2]{\frac{d #1}{d #2}}

% Packages
\usepackage{amsmath}
\usepackage[margin=0.75in]{geometry}
\usepackage{lipsum}
\usepackage{tikz}
\usepackage{pgfplots}
\usepackage{graphicx}
\usepackage{amssymb}
\usepackage{listings}


% Document
\begin{document}
    \maketitle
    \begin{enumerate}
        \item
        \begin{enumerate}
            \item[i] These constraints can be found by defining the equations for the position of each of the elements and then solving for their velocities.
            Defining the center of the first wheel as $(x_1,y_1)$ and the second as $(x_2,y_2)$ we get:\footnote{https://physics.stackexchange.com/questions/105864/is-there-a-systematic-way-to-derive-constraint-equations}
            \begin{gather*}
                x_1=x-\frac{l}{2}\cos\theta\quad y_1=y-\frac{l}{2}\sin\theta\\
                x_2=x+\frac{l}{2}\cos\theta\quad y_2=y+\frac{l}{2}\sin\theta\\
            \end{gather*}
            Taking the derivatives:
            \begin{gather*}
                \dot{x}_1=\dot{x}+\frac{l}{2}\dot{\theta}\sin\theta\quad \dot{y}_1=\dot{y}-\frac{l}{2}\dot{\theta}\cos\theta\\
                \dot{x}_2=\dot{x}-\frac{l}{2}\dot{\theta}\sin\theta\quad \dot{y}_2=\dot{y}+\frac{l}{2}\dot{\theta}\cos\theta
            \end{gather*}
            We also know the velocities of the center of each wheel is constrained by the velocity of the wheel:
            \begin{gather*}
                \dot{x}_1=R\dot{\phi}\sin\theta\quad \dot{y}_1=-R\dot{\phi}\cos\theta\\
                \dot{x}_2=R\dot{\phi}'\sin\theta\quad \dot{y}_2=-R\dot{\phi}'\cos\theta
            \end{gather*}
            Pretending $dt$ is just part of a fraction to piss off mathematicians:
            \begin{gather*}
                dx_1=dx+\frac{l}{2}d\theta\sin\theta\quad dy_1=dy-\frac{l}{2}d\theta\cos\theta\\
                dx_2=dx-\frac{l}{2}d\theta\sin\theta\quad dy_2=dy+\frac{l}{2}d\theta\cos\theta\\
                dx_1=Rd\phi\sin\theta\quad dy_1=-Rd\phi\cos\theta\\
                dx_2=Rd\phi'\sin\theta\quad dy_2=-Rd\phi'\cos\theta
            \end{gather*}
            Substituting the different equations to solve for $dx$ and $dy$:
            \begin{gather*}
                dx+\frac{l}{2}d\theta\sin\theta=Rd\phi\sin\theta\\
                dx=\sin\theta\left(Rd\phi-\frac{l}{2}d\theta\right)\\
                dy-\frac{l}{2}d\theta\cos\theta=-Rd\phi\cos\theta\\
                dy=\cos\theta\left(\frac{l}{2}d\theta-Rd\phi\right)
            \end{gather*}
            The first equation is easy to see from this by substituting for $dx$ and $dy$:
            \begin{gather*}
                \sin\theta dy=-\cos\theta dx\\
                \sin\theta\left[\cos\theta\left(\frac{l}{2}d\theta-Rd\phi\right)\right]=\cos\theta\left[-\sin\theta\left(Rd\phi-\frac{l}{2}d\theta\right)\right]\\
                \sin\theta\left[\cos\theta\left(\frac{l}{2}d\theta-Rd\phi\right)\right]=\cos\theta\left[\sin\theta\left(\frac{l}{2}d\theta-Rd\phi\right)\right]
            \end{gather*}
            We can derive the second equation using the equations for $dx$ and $dy$ using the midpoint:
            \begin{gather*}
                dx=\frac{dx_2+dx_1}{2}\quad dy=\frac{dy_2+dy_2}{2}\\
                dx=\frac{R\sin\theta(d\phi+d\phi')}{2}\quad dy=-\frac{R\cos\theta(d\phi+d\phi')}{2}\\
                \sin\theta dx-\cos\theta dy=R(d\phi+d\phi')\\
                \sin\theta \frac{R\sin\theta(d\phi+d\phi')}{2}+\cos\theta\frac{R\cos\theta(d\phi+d\phi')}{2}=R(d\phi+d\phi')\\
                \frac{R(d\phi+d\phi')}{2}+\frac{R(d\phi+d\phi')}{2}=R(d\phi+d\phi')\\
                R(d\phi+d\phi')=R(d\phi+d\phi')\\
            \end{gather*}
            \item[ii]
            This constraint states that the derivatives must be equal to 0:
            \begin{gather*}
                \frac{d}{dt}\left(\theta+\frac{R}{l}(\phi-\phi')\right)=d\theta+\frac{R}{l}(d\phi-d\phi')=0\\
                d\theta=-\frac{R}{l}(d\phi-d\phi')\\
            \end{gather*}
            We also know that the general formula for $d\theta$ of a vector is:
            \begin{gather*}
                d\theta=\frac{xdy-ydx}{x^2+y^2}\\
            \end{gather*}
            But since this is the relative vector, and the distance is constant:
            \begin{gather*}
                d\theta=\frac{l\cos\theta dy-l\sin\theta dx}{l}\\
                d\theta=\cos\theta dy-\sin\theta dx
            \end{gather*}
            Plugging in the various variables:
            \begin{gather*}
                d\theta=-\cos\theta \cos\theta\left(\frac{l}{2}d\theta+Rd\phi\right)-\sin\theta \sin\theta\left(Rd\phi+\frac{l}{2}d\theta\right)\\
                d\theta=-\cos^2\theta\left(\frac{l}{2}d\theta'+Rd\phi\right)- \sin^2\theta\left(Rd\phi+\frac{l}{2}d\theta\right)\\
                d\theta=-\frac{R}{l}(d\phi-d\phi')
            \end{gather*}
        \end{enumerate}
        \item
        \begin{enumerate}
            \item
            Using d'Alembert's principle
            \begin{gather*}
                ((T-mg)-m\ddot{y}_m)\delta y_m=0\\
                ((T-Mg)-(-M\ddot{y}_m))\delta y_M=0
            \end{gather*}
            \item
            Since we know:
            \begin{gather*}
                \delta y_m=-\delta y_M
            \end{gather*}
            Solving that equation
            \begin{gather*}
                -m\ddot{y}_m+(T-mg)=(T-Mg)-(-M\ddot{y}_m)\\
                -m\ddot{y}_m-mg=-Mg-(-M\ddot{y}_m)\\
                -m\ddot{y}_m-M\ddot{y}_m=-Mg+mg\\
                \ddot{y}_m=\frac{M-m}{M+m}g\\
            \end{gather*}
            \item
            Using the same steps above:
            \begin{gather*}
                ((T-mg\sin\beta)-(-m\ddot{L}_m))\delta y_m=0\\
                ((T-Mg\sin\alpha)-M\ddot{L}_m)\delta y_M=0
            \end{gather*}
            Solving that equation:
            \begin{gather*}
                -m\ddot{L}_m+(T-(-mg\sin\beta))=(T-Mg\sin\alpha)-M\ddot{L}_m\\
                -m\ddot{L}_m+mg\sin\beta=-Mg\sin\alpha-M\ddot{L}_m\\
                -m\ddot{L}_m-M\ddot{L}_m=Mg\sin\alpha=mg\sin\beta\\
                \ddot{L}_m=\frac{m\sin\alpha-M\sin\beta}{M+m}g\\
            \end{gather*}
            \item
            Using d'Alembert's principle, and specifying the distance between the first pulley and $m_1$ as $y_1$, the distance between the first pulley and the second pulley as $y_p$, the distance between the second pulley and $m_2$ as $y_2$, and the distance between the second pulley and $m_3$ as $y_3$:
            \begin{gather*}
                ((T_1-m_1 g)+m_1\ddot{y}_1)\delta y_1=0\\
                ((T_2-m_2 g)+m_2\ddot{y}_2)\delta y_2=0\\
                ((T_2-m_3 g)-m_3\ddot{y}_3)\delta y_3=0
            \end{gather*}
            We also know the following constraints:
            \begin{gather*}
                y_1+y_p=l\\
                y_2+y_3=L\\
                \delta y_1=-\delta y_p\\
                \delta y_p=\frac{\delta y_2+\delta y_3}{2}
            \end{gather*}
            \item
            Re-arranging the equations form part (d):
            \begin{gather*}
                \ddot{y}_1=-\frac{T_1-m_1}{m_1}\\
                \ddot{y}_2=-\frac{T_2-m_2}{m_2}\\
                \ddot{y}_3=\frac{T_2-m_3}{m_3}
            \end{gather*}
            \item
            Starting with d'Alembert's principle and doing algebra:
            \begin{gather*}
                (m_1 (g-\ddot{y}_1))\delta y_1+(m_2 (g-\ddot{y}_2))\delta y_2+(m_3 (g-\ddot{y}_3))\delta y_3=0\\
                -(m_1 (g-\ddot{y}_1))\frac{\delta y_2+\delta y_3}{2}+(m_2 (g-\ddot{y}_2))\delta y_2+(m_3 (g-\ddot{y}_3))\delta y_3=0\\
            \end{gather*}
            We can split up the $\delta y_3$ and $\delta y_2$ terms because they both independently have to be 0:
            \begin{gather*}
                \frac{-(m_1 (g-\ddot{y}_1))}{2}+(m_2 (g-\ddot{y}_2))=0\\
                \frac{-(m_1 (g-\ddot{y}_1))}{2}+(m_3 (g-\ddot{y}_3))=0\\
            \end{gather*}
            Since $\delta y_p=\frac{\delta y_2+\delta y_3}{2}$ :
            \begin{gather*}
                \frac{-(m_1 (g+\frac{\ddot{y_2}+\ddot{y_3}}{2}))}{2}+(m_2 (g-\ddot{y}_2))=0\\
                \frac{-(m_1 (g+\frac{\ddot{y_2}+\ddot{y_3}}{2}))}{2}+(m_3 (g-\ddot{y}_3))=0\\
            \end{gather*}
            Using Mathematica:\\
            \includegraphics[width=\linewidth]{HW_3_scripts/mathematica_screenshot}\\
            \begin{gather*}
                m_3\left(g-y_3\right)= m_2\left(g-y_2\right)
            \end{gather*}
            This equation would be stationary if the masses equal each other:
            \begin{gather*}
                m_2=\frac{m_3\left(g-y_3\right)}{\left(g-y_2\right)}\\
                m_3= \frac{m_2\left(g-y_2\right)}{\left(g-y_3\right)}
            \end{gather*}
        \end{enumerate}
        \item
        \begin{enumerate}
            \item
            Using cylindrical coordinates the Lagrangian is:
            \begin{gather*}
                L=\frac{1}{2}m\left(\dot{\rho}^2+\rho^2\dot{\phi}^2\right)+\frac{1}{2}M\dot{z}^2-Mgz
            \end{gather*}
            We can impose the constraints:
            \begin{gather*}
                \dot{\rho}=-\dot{z}\\
                z=l-\rho
            \end{gather*}
            This gives us the following equation:
            \begin{gather*}
                L=\frac{1}{2}m\left(\dot{\rho}^2+\rho^2\dot{\phi}^2\right)+\frac{1}{2}M\dot{\rho}^2-Mg(l-\rho)
            \end{gather*}
            Solving for the equations of motion in terms of $\rho$:
            \begin{gather*}
                \frac{d}{dt}\left(\p{L}{\dot{\rho}}\right)-\der{L}{\rho}=0\\
                \frac{d}{dt}\left(\p{L}{\dot{\rho}}\right)=(m+M)\ddot{\rho}\\
                \der{L}{\rho}=m\rho\dot{\phi}^2+Mg\\
                (m+M)\ddot{\rho}-m\rho\dot{\phi}^2-Mg=0\\
                \rho(t)=\rho_0 e^{\sqrt{\frac{m}{m+m}}\dot{\phi} t}+\rho_1 e^{-\sqrt{\frac{m}{m+m}}\dot{\phi} t}-\frac{gM}{m\dot{\phi}^2}
            \end{gather*}
            Solving for the equations of motion in terms of $\phi$:
            \begin{gather*}
                \frac{d}{dt}\left(\p{L}{\dot{\phi}}\right)-\der{L}{\phi}=0\\
                \frac{d}{dt}\left(\p{L}{\dot{\phi}}\right)=m\rho^2\ddot{\phi}\\
                \der{L}{\phi}=0\\
                m\rho^2\ddot{\phi}=0\\
                \phi(t)=\phi_1 t+\phi_0
            \end{gather*}
            \item
            The significance of these equations is that the rotation of the block on the table does not change.
            The distance of the block from the hole does change however, and is even influenced by the rotation of the block.
            The energy, as well as the angular acceleration, are conserved quantities.
        \end{enumerate}
        \item
        \begin{enumerate}
            \item
            The situation in the problem can be described by:
            \begin{gather*}
                \ddot{x}=-x
            \end{gather*}
            Solving this equation gives:
            \begin{gather*}
                x(t)=c_1\sin(t)+c_2\cos(t)\\
                \dot{x}(t)=c_1\cos(t)-c_2\sin(t)
            \end{gather*}
            This results in the following phase portrait:\\
            \linebreak
            \begin{center}
                \includegraphics[width=0.75\linewidth]{HW_3_scripts/phase_portrait}\\
                \includegraphics[width=0.75\linewidth]{HW_3_scripts/phase_portrait_code}
            \end{center}
            \item Each phase portrait is a circle, so it's radius will never change, so if it begins in the are it will stay in the area.
            \item
            Starting with the energy equation:
            \begin{gather*}
                E=\frac{p^2}{2m}+mgq\\
            \end{gather*}
            Solving for $q$:
            \begin{gather*}
                q=\frac{1}{mg}\left[E-\frac{p^2}{2m}\right]\\
                q_1=\frac{1}{mg}\left[E'-\frac{p^2}{2m}\right]\\
                q_2=\frac{1}{mg}\left[E''-\frac{p^2}{2m}\right]\\
            \end{gather*}
            Integrating to find the area:
            \begin{gather*}
                \int_{p_1}^{p_2}\int_{q(p_1)}^{q(p_2)}dqdp\\
                \int_{p_1}^{p_2}\int_{\frac{1}{mg}\left[E'-\frac{p^2}{2m}\right]}^{\frac{1}{mg}\left[E''-\frac{p^2}{2m}\right]}dqdp\\
                \left.\int_{p_1}^{p_2}\right|_{\frac{1}{mg}\left[E'-\frac{p^2}{2m}\right]}^{\frac{1}{mg}\left[E''-\frac{p^2}{2m}\right]}dp\\
                \left.\int_{p_1}^{p_2}\frac{1}{mg}\left[E'-\frac{p^2}{2m}\right]-\frac{1}{mg}\left[E''-\frac{p^2}{2m}\right]dp\\
                \frac{1}{mg}(E'-E'')\left.\int_{p_1}^{p_2}dp\\
                A=\frac{1}{mg}(E'-E'')(p_2-p_1)\\
            \end{gather*}
            \item
            Starting with the Lagrangian to solve the equations of motion:
            \begin{gather*}
                L=\frac{1}{2}m\dot{q}^2-mgq\\
                \frac{d}{dt}\left(\p{L}{\dot{q}}\right)-\der{L}{q}=0\\
                \frac{d}{dt}\left(\p{L}{\dot{q}}\right)=m\ddot{q}\\
                \der{L}{q}=-mg\\
                m\ddot{q}+mg=0\\
                q(t)=c_2 t+c_1-\frac{gt^2}{2}
            \end{gather*}
            Integrating $q(t)$ to get the momentum:
            \begin{gather*}
                \dot{q}(t)=\int\left( c_2 t+c_1-\frac{gt^2}{4} \right) dt=c_2-gt
            \end{gather*}
            Plugging this back into the equation for $A$:
            \begin{gather*}
                A=\frac{1}{mg}(E'-E'')(p_2-p_1)\\
                A=\frac{1}{mg}(E'-E'')(c_1-gt-c_2+gt)\\
                A=\frac{1}{mg}(E'-E'')(c_1-c_2)\\
            \end{gather*}
            Since we already know $E'$ and $E''$ are constant, and the rest of the equation is constant $A$ is constant.
        \end{enumerate}
    \end{enumerate}

\end{document}