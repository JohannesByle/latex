%! Author = Johannes Byle
%! Date = 10/2/2021

% Preamble
\documentclass[12pt]{article}
\title{Math Methods Assignment \#5}
\author{Johannes Byle}
\newcommand{\p}[2]{\frac{\partial #1}{\partial #2}}
\newcommand{\der}[2]{\frac{d #1}{d #2}}

% Packages
\usepackage{amsmath}
\usepackage[margin=0.75in]{geometry}
\usepackage{lipsum}
\usepackage{physics}
\usepackage{bbold}

% Document
\begin{document}
  \maketitle
  \begin{enumerate}
    \item
    \begin{enumerate}
      \item From the definition of orthogonality given by Arfken (page 105), to show that they are orthogonal we must show that $g_{ij}=\p{\pmb{r}}{q_i}\cdot\p{\pmb{r}}{q_j}=0$ for all $i\neq j$:
      \begin{gather*}
        \p{\pmb{r}}{q_{\eta}}\cdot\p{\pmb{r}}{q_{\phi}}=
        \left(\sinh\eta\cos\phi\hat{i}+\cosh\eta\sin\phi\hat{j}\right)\cdot
        \left(-\cosh\eta\sin\phi\hat{i}+\sinh\eta\cos\phi\hat{j}\right)=0\\
        \p{\pmb{r}}{q_{\eta}}\cdot\p{\pmb{r}}{q_z}=
        \left(\sinh\eta\cos\phi\hat{i}+\cosh\eta\sin\phi\hat{j}\right)\cdot
        \left(\hat{k}\right)=0\\
        \p{\pmb{r}}{q_{\phi}}\cdot\p{\pmb{r}}{q_z}=
        \left(-\cosh\eta\sin\phi\hat{i}+\sinh\eta\cos\phi\hat{j}\right)\cdot
        \left(\hat{k}\right)=0
      \end{gather*}
      \item The metric tensor is defined as $ds^2=g_{ij}dq_i dq_j$.
      Since we know the coordinate system is orthogonal we only need to calculate the diagonal terms:
      \begin{gather*}
        ds^2=
        \begin{bmatrix}
          d\eta & d\phi & dz
        \end{bmatrix}
        \begin{bmatrix}
          \sinh^2\eta\cos^2\phi+\cosh^2\eta\sin^2\phi & 0                                           & 0 \\
          0                                           & \cosh^2\eta\sin^2\phi+\sinh^2\eta\cos^2\phi & 0 \\
          0                                           & 0                                           & 1
        \end{bmatrix}
        \begin{bmatrix}
          d\eta \\
          d\phi \\
          dz
        \end{bmatrix}
      \end{gather*}
      \item Defining $g_{ii}=h_i^2$ we can use the relation from Arfken: $\nabla f=\hat{q}_i\frac{1}{h_i}\p{f}{q_i}$:
      \begin{gather*}
        \nabla f=\frac{\hat{\eta}}{\sinh\eta\cos\phi+\cosh\eta\sin\phi}\p{f}{\eta}+
        \frac{\hat{\phi}}{\cosh\eta\sin\phi+\sinh\eta\cos\phi}\p{f}{\phi}+\hat{z}\p{f}{z}
      \end{gather*}
      \item Using equation (2.21) from Arfken:
      \begin{gather*}
        \nabla\cdot\mathcal{E}=\frac{\p{}{\eta}\mathcal{E}_{\eta}+\p{}{\phi}\mathcal{E}_{\phi}+\p{}{z}\mathcal{E}_{z}(\cosh\eta\sin\phi+\sinh\eta\cos\phi)}{(\cosh\eta\sin\phi+\sinh\eta\cos\phi)}
      \end{gather*}
      \item Using equation (2.22) from Arfken:
      \begin{gather*}
        \nabla^2\cdot\mathcal{E}=\frac{\p{^2f}{\eta^2}+\p{^2f}{\phi^2}+\p{}{z}\left[(\cosh\eta\sin\phi+\sinh\eta\cos\phi)^2\p{f}{z}\right]}{(\cosh\eta\sin\phi+\sinh\eta\cos\phi)^2}
      \end{gather*}
    \end{enumerate}
    \item
    \begin{gather*}
      \nabla^2\phi=\frac{\epsilon^4-6\epsilon^2 r^2+r^4}{r(\epsilon^2+r^2)^3}
    \end{gather*}
    \begin{enumerate}
      \item Looking at the limits in steps:
      \begin{align*}
        \lim_{\epsilon\rightarrow0}\nabla^2\phi(r_0)=\frac{r_0^4}{r_0^6}=\frac{1}{r_0^2}
      \end{align*}
      This diverges as $r_0$ approaches 0:
      \begin{align*}
        \lim_{r_0\rightarrow0}\frac{1}{r_0^2}\rightarrow\infty
      \end{align*}
      \item

    \end{enumerate}
  \end{enumerate}
\end{document}