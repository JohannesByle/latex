%! Author = Johannes Byle
%! Date = 10/2/2021

% Preamble
\documentclass[12pt]{article}
\title{Math Methods Assignment \#5}
\author{Johannes Byle}
\newcommand{\p}[2]{\frac{\partial #1}{\partial #2}}
\newcommand{\der}[2]{\frac{d #1}{d #2}}

% Packages
\usepackage{amsmath}
\usepackage[margin=0.75in]{geometry}
\usepackage{lipsum}
\usepackage{physics}
\usepackage{bbold}

% Document
\begin{document}
  \maketitle
  \begin{enumerate}
    \item
    \begin{enumerate}
      \item From the definition of orthogonality given by Arfken (page 105), to show that they are orthogonal we must show that $g_{ij}=\p{\pmb{r}}{q_i}\cdot\p{\pmb{r}}{q_j}=0$ for all $i\neq j$:
      \begin{gather*}
        \p{\pmb{r}}{q_{\eta}}\cdot\p{\pmb{r}}{q_{\phi}}=
        \left(\sinh\eta\cos\phi\hat{i}+\cosh\eta\sin\phi\hat{j}\right)\cdot
        \left(-\cosh\eta\sin\phi\hat{i}+\sinh\eta\cos\phi\hat{j}\right)=0\\
        \p{\pmb{r}}{q_{\eta}}\cdot\p{\pmb{r}}{q_z}=
        \left(\sinh\eta\cos\phi\hat{i}+\cosh\eta\sin\phi\hat{j}\right)\cdot
        \left(\hat{k}\right)=0\\
        \p{\pmb{r}}{q_{\phi}}\cdot\p{\pmb{r}}{q_z}=
        \left(-\cosh\eta\sin\phi\hat{i}+\sinh\eta\cos\phi\hat{j}\right)\cdot
        \left(\hat{k}\right)=0
      \end{gather*}
      \item The metric tensor is defined as $ds^2=g_{ij}dq_i dq_j$.
      Since we know the coordinate system is orthogonal we only need to calculate the diagonal terms:
      \begin{gather*}
        ds^2=
        \begin{bmatrix}
          d\eta & d\phi & dz
        \end{bmatrix}
        \begin{bmatrix}
          \sinh^2\eta\cos^2\phi+\cosh^2\eta\sin^2\phi & 0                                           & 0 \\
          0                                           & \cosh^2\eta\sin^2\phi+\sinh^2\eta\cos^2\phi & 0 \\
          0                                           & 0                                           & 1
        \end{bmatrix}
        \begin{bmatrix}
          d\eta \\
          d\phi \\
          dz
        \end{bmatrix}
      \end{gather*}
      \item Defining $g_{ii}=h_i^2$ we can use the relation from Arfken: $\nabla f=\hat{q}_i\frac{1}{h_i}\p{f}{q_i}$:
      \begin{gather*}
        \nabla f=\frac{\hat{\eta}}{\sinh\eta\cos\phi+\cosh\eta\sin\phi}\p{f}{\eta}+
        \frac{\hat{\phi}}{\cosh\eta\sin\phi+\sinh\eta\cos\phi}\p{f}{\phi}+\hat{z}\p{f}{z}
      \end{gather*}
      \item Using equation (2.21) from Arfken:
      \begin{gather*}
        \nabla\cdot\mathcal{E}=\frac{\p{}{\eta}\mathcal{E}_{\eta}+\p{}{\phi}\mathcal{E}_{\phi}+\p{}{z}\mathcal{E}_{z}(\cosh\eta\sin\phi+\sinh\eta\cos\phi)}{(\cosh\eta\sin\phi+\sinh\eta\cos\phi)}
      \end{gather*}
      \item Using equation (2.22) from Arfken:
      \begin{gather*}
        \nabla^2\cdot\mathcal{E}=\frac{\p{^2f}{\eta^2}+\p{^2f}{\phi^2}+\p{}{z}\left[(\cosh\eta\sin\phi+\sinh\eta\cos\phi)^2\p{f}{z}\right]}{(\cosh\eta\sin\phi+\sinh\eta\cos\phi)^2}
      \end{gather*}
    \end{enumerate}
    \item
    \begin{gather*}
      \nabla^2\phi=\frac{2 \epsilon ^4-6 r^2 \epsilon ^2}{r \left(r^2+\epsilon ^2\right)^3}
    \end{gather*}
    \begin{enumerate}
      \item Looking at the limits in steps:
      \begin{align*}
        \lim_{\epsilon\rightarrow0}\nabla^2\phi(r_0)=\frac{0}{r_0 \left(r_0^2\right)^3}
      \end{align*}
      \begin{align*}
        \lim_{r_0\rightarrow0}0\rightarrow0
      \end{align*}
      \item Again, looking at the limits in steps:
      \begin{align*}
        \lim_{\epsilon\rightarrow0}\nabla^2\phi(r_0)=\frac{2\epsilon^4}{0}\rightarrow\infty
      \end{align*}
      \begin{align*}
        \lim_{r_0\rightarrow0}\infty\rightarrow\infty
      \end{align*}
      \item At all points other than zero we know from part (a) that when $\epsilon\rightarrow0$ $\phi(r)\rightarrow\frac{1}{r}$ the limit of the laplacian approaches 0.
      But for the point at zero, we can treat $\epsilon$ as a number infinitely close to the origin, since we have already evaluated $\phi(r)$ at $r=r_0\rightarrow\infty$ which as we approach the origin ($\epsilon\rightarrow0$).
      Since $\nabla^2\frac{1}{r}$ is zero except for the origin, it is a function that has the same behaviour as the delta function, and thus $\int f(r)\nabla^2\frac{1}{r}d^3r=f(0)$.
    \end{enumerate}
    \item
    \begin{enumerate}
      \item Starting from the definition of polar coordinates:
      \begin{align*}
        x&=\rho\cos\varphi\\
        y&=\rho\sin\varphi
      \end{align*}
      \begin{align*}
        \nabla f&=\frac{\hat{\varphi}}{\p{\pmb{r}}{\varphi}\cdot\p{\pmb{r}}{\varphi}}\p{f}{\varphi}+\frac{\hat{\rho}}{\p{\pmb{r}}{\rho}\cdot\p{\pmb{r}}{\rho}}\p{f}{\rho}\\
        \nabla f&=\frac{\hat{\varphi}}{\rho\left(\cos^2\varphi+\sin^2\varphi \right)}\p{f}{\varphi}+\frac{\hat{\rho}}{\cos^2\varphi+\sin^2\varphi}\p{f}{\rho}\\
        \nabla f&=\frac{\hat{\varphi}}{\rho}\p{f}{\varphi}+\hat{\rho}\p{f}{\rho}
      \end{align*}
      \item
      \begin{align*}
        \nabla\times\vec{g}&=\p{}{\rho}\frac{\hat{\varphi}}{\rho}\p{f}{\varphi}+\p{}{\varphi}\hat{\rho}\p{f}{\rho}\\
        \nabla\times\vec{g}&=\left(-\frac{1}{\rho^2}\p{f}{\varphi}+\frac{1}{\rho}\p{}{\rho}\p{f}{\varphi}\right)\hat{\varphi}+\hat{\rho}\p{}{\varphi}\p{f}{\rho}
      \end{align*}
      \item Since the gradient of scalar field is conservative, the line integral will be zero, since it's a loop and will start and end in the same points.\footnote{https://math.stackexchange.com/questions/1435044/line-integral-of-conservative-field-in-polar-coordinates}
      \item This agrees with the question statement.
    \end{enumerate}
  \end{enumerate}
\end{document}