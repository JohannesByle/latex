% Preamble
\documentclass[12pt]{article}
\title{Math Methods Assignment \#4}
\author{Johannes Byle}
\newcommand{\p}[2]{\frac{\partial #1}{\partial #2}}
\newcommand{\der}[2]{\frac{d #1}{d #2}}

% Packages
\usepackage{amsmath}
\usepackage[margin=0.75in]{geometry}
\usepackage{lipsum}
\usepackage{physics}

% Document
\begin{document}
  \maketitle
  \begin{enumerate}
    \item It would depend on whether the rocket rotated.
    Assuming that it's acceleration was constant in both the horizontal and vertical directions it would move in a parabola, unless $v_0$ is 0 in which case it would move in a straight line.
    If the rocket were allowed to rotate it would move in a parabolic path, as it would start asymptotically in horizontal direction, and then as it tilted down it would move asymptotically close to the vertical axis.
    \item If we take a rhombus with corners $A,\ B,\ C,\ D$ we can represent the sides as $\vec{AB},\ \vec{BC},\ \vec{CD},\ \vec{DA}$.
    We know that $|\vec{AB}|=|\vec{BC}|=|\vec{CD}|=|\vec{DA}|$, and that $\vec{AC}=\vec{AB}+\vec{BC}$ and $\vec{BD}=\vec{BC}+\vec{CD}$.
    Since the sides of a rhombus are parallel we can for the sake of this problem consider $\vec{AC}=\vec{AB}+\vec{AD}$ and $\vec{BD}=\vec{AD}-\vec{AB}$.
    To show that the diagonals are orthogonal we need to show that the dot products are equal to zero: $\vec{AC}\cdot\vec{BD}=\left(\vec{AB}+\vec{AD}\right)\cdot\left(\vec{AD}-\vec{AB}\right)=\vec{AD}^2-\vec{AB}^2=0$ since the sides all have equal length.
    \item This is equivalent to $\sum_i\sum_j\sum_k \epsilon_{ijk}\epsilon_{ijk}$.
    The number of permutations of $n$ numbers is $n!$, which means there are $3!=6$ permutations in our case.
    Since whether the permutation is even or odd the term $\epsilon_{ijk}\epsilon_{ijk}=1$ and otherwise 0 the sum is equal to 6.
    \item Using the definition of the cross product $\pmb{a}\times\pmb{b}=\epsilon_{ijk}\pmb{\hat{e}}_i a_j b_k$:
    \begin{gather*}
      \pmb{a}\times(\pmb{b}\times\pmb{c})=\pmb{a}\times(\epsilon_{ijk}\pmb{\hat{e}}_i b_j c_k)=\epsilon_{ijk}\pmb{\hat{e}}_i a_j b_j c_k\\
      (\pmb{a}\times\pmb{b})\times\pmb{c}=(\epsilon_{ijk}\pmb{\hat{e}}_i a_j b_k)\times\pmb{c}=\epsilon_{ijk}\pmb{\hat{e}}_i a_j b_j c_k\\
    \end{gather*}
    \item
  \end{enumerate}

\end{document}