% Preamble
\documentclass[12pt]{article}
\title{Math Methods Assignment \#4}
\author{Johannes Byle}
\newcommand{\p}[2]{\frac{\partial #1}{\partial #2}}
\newcommand{\der}[2]{\frac{d #1}{d #2}}

% Packages
\usepackage{amsmath}
\usepackage[margin=0.75in]{geometry}
\usepackage{lipsum}
\usepackage{physics}

% Document
\begin{document}
  \maketitle
  \begin{enumerate}
    \item It would depend on whether the rocket rotated.
    Assuming that it's acceleration was constant in both the horizontal and vertical directions it would move in a parabola, unless $v_0$ is 0 in which case it would move in a straight line.
    If the rocket were allowed to rotate it would move in a parabolic path, as it would start asymptotically in horizontal direction, and then as it tilted down it would move asymptotically close to the vertical axis.
    \item
  \end{enumerate}

\end{document}