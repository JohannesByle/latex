%! Author = Johannes Byle
%! Date = 8/28/2021

% Preamble
\documentclass[12pt]{article}
\title{Math Methods Assignment \#1}
\author{Johannes Byle}

% Packages
\usepackage{amsmath}
\usepackage[margin=0.75in]{geometry}
\usepackage{lipsum}

% Document
\begin{document}
    \maketitle
    \begin{enumerate}
        \item
        \begin{enumerate}
            \item
            \begin{gather*}
                E=\frac{1}{2}mv^2\\
                P_0=\frac{E}{\Delta t}=\frac{\Delta mv^2}{2\Delta t}\\
                v=\sqrt{\frac{2P_0\Delta t}{\Delta m}}\\
                p=\Delta m\sqrt{\frac{2P_0\Delta t}{\Delta m}}=\sqrt{2P_0\Delta m\Delta t}
            \end{gather*}
            \item
            Tsiolkovsky rocket equation from wikipedia:
            \[
                \Delta v=v_e\ln\frac{m_0}{m_f}
            \]
            Substituting in the answer from part (a):
            \[
                v_f=\sqrt{2P_0\frac{\Delta t}{\Delta m}}\ln\left(\frac{m}{m-\Delta m}\right)
            \]
            \item
            From the same wikipedia article, substituting in the answer from part (a) and converting $\Delta$ to $d$:
            \begin{gather*}
                mv'=-v_e m'\\
                mv'+v_e m'=0\\
                v'=\sqrt{2P_0 \frac{dt}{dm}}\frac{m'}{m}
            \end{gather*}
            Assuming $\frac{dt}{dm}=(m')^{-1}$:
            \begin{gather*}
                v'=\sqrt{\frac{2P_0}{m'}}\frac{m'}{m}\\
                v=\sqrt{2P_0}\int_0^{t_f}\sqrt{\frac{1}{m'}}\frac{m'}{m}dt
            \end{gather*}
            Using the Euler-Lagrange equation:
            \begin{gather*}
                m'\frac{\delta f}{\delta m'}-f=c\\
                m'\frac{\delta}{\delta m'}\left(\sqrt{\frac{1}{m'}}\frac{m'}{m}\right)-\sqrt{\frac{1}{m'}}\frac{m'}{m}=c\\
                m'\frac{1}{2 m}\sqrt{\frac{1}{m'}}-\sqrt{\frac{1}{m'}}\frac{m'}{m}=c\\
                -\frac{m'}{2m}\sqrt{\frac{1}{m'}}=c\\
                m'=4c^2 m^2
            \end{gather*}
            \item
            Final velocity in part (c), using the equation for v derived in the first part of (c):
            \begin{gather*}
                v=\sqrt{2P_0}\int_0^{t_f}\sqrt{\frac{1}{m'}}\frac{m'}{m}dt\\
                v=\sqrt{2P_0}\int_0^{t_f}\sqrt{\frac{1}{4c^2 m^2}}\frac{4c^2 m^2}{m}dt\\
                v=\sqrt{2P_0}\int_0^{t_f}2 c^2 m \sqrt{\frac{1}{c^2 m^2}}dt\\
                v=2c^2 m\sqrt{\frac{2P_0}{c^2 m^2}}t_f
            \end{gather*}
            If $c$ is positive:
            \begin{gather*}
                v=2ct_f\sqrt{2P_0}
            \end{gather*}
            This differs from the answer in part (b) because it does not depend on $m'$ or $m$?
        \end{enumerate}
        \item
        \begin{enumerate}
            \item
            Using the Euler-Lagrange equation:
            \begin{gather*}
                y'\frac{\delta f}{\delta y'}-f=c\\
                y'\frac{\delta}{\delta y'}\left[\left(\frac{\delta y}{\delta x}\right)^2+\alpha y\frac{\delta y}{\delta x}\right]-\left[\left(\frac{\delta y}{\delta x}\right)^2+\alpha y\frac{\delta y}{\delta x}\right]=c\\
                y'\frac{\delta}{\delta y'}\left[\left(y'\right)^2+\alpha yy'\right]-\left[\left(y'\right)^2+\alpha yy'\right]=c\\
                y'\cdot\left(2y'+\alpha y\right)-\left[\left(y'\right)^2+\alpha yy'\right]=c\\
                y'=\sqrt{c}\\
            \end{gather*}
            Since $y'$ is constant:
            \[
                y=Ax+B
            \]
            Since $y(0)=0$ and $y(x_f)=y_f$:
            \begin{gather*}
                B=0\\
                y_f=A\cdot x_f
                A=\frac{y_f}{x_f}
            \end{gather*}
            \item It doesn't because $\alpha y\frac{\delta y}{\delta x}$ is a total derivative and thus is path independent.

        \end{enumerate}
        \item
        Differentiating $I(\epsilon)$ with respect to $\epsilon$:
        \[
            \frac{dI}{d\epsilon}=\int_{x_A}^{x_B}\left[\frac{\delta f}{\delta y}\frac{dy}{d\epsilon}+\frac{\delta f}{\delta y'}\frac{dy'}{d\epsilon}+\frac{\delta f}{\delta y''}\frac{dy''}{d\epsilon}\right]
        \]
        Since $y'$ endpoints are prescribed, the integration by parts trick on page 46 will work on both the $\frac{\delta f}{\delta y'}\frac{dy'}{d\epsilon}$ and $\frac{\delta f}{\delta y''}\frac{dy''}{d\epsilon}$ terms.
        \[
            \frac{dI}{d\epsilon}=\int_{x_A}^{x_B}\left[\frac{\delta f}{\delta y}-\frac{d}{dx}\left(\frac{\delta f}{\delta y'}\right)-\frac{d^2}{dx^2}\left(\frac{\delta f}{\delta y'}\right)\right]\frac{dy}{d\epsilon}dx
        \]
        This requires that $y(x,\epsilon)$ and all its derivatives through third order are continuous functions of $x$ and $\epsilon$
        \item
        \begin{enumerate}
            \item
            From the definition of $ds$:
            \[
                ds=\sqrt{1+y'^2}dx
            \]
            Using the Euler-Lagrange equation:
            \begin{gather*}
                f=\frac{\sqrt{1+y'^2}}{u}\\
                y'\frac{\delta}{\delta y'}\left[\frac{\sqrt{1+y'^2}}{u}\right]-\frac{\sqrt{1+y'^2}}{u}=\text{const}\\
                \frac{y'^2}{u\sqrt{y'^2+1}}-\frac{\sqrt{1+y'^2}}{u}=\frac{y'^2-\left(1+y'^2\right)}{u\sqrt{y'^2+1}}=\text{const}\\
                \frac{1}{u\sqrt{y'^2+1}}=\frac{1}{u\sqrt{\left(\frac{dy}{dx}\right)^2+1}}=\frac{1}{u\sqrt{\cot^2{\phi}+1}}=\text{const}\\
                \frac{1}{u\sqrt{\csc^2\phi}}=\frac{\sin\phi}{u}=\text{const}
            \end{gather*}
            \item
            Since we assumed $u$ above was a function of $y$ we can simply take our result from part (a):
            \begin{gather*}
                \frac{\sin\phi}{\alpha y}=\text{const}\\
                \sin\phi=\alpha y\\
                \sin\left(\arctan y'\right)=\alpha y
            \end{gather*}
            Using WolframAlpha to solve this differential equation gives:
            \begin{gather*}
                \alpha x+x_0=\sqrt{1-\alpha^2y^2}-\tanh^{-1}\left(\sqrt{1-\alpha^2 y^2}\right)
            \end{gather*}
            Assuming the $\tanh^{-1}$ can be ignored as a phase or something similar, it's clear that $\alpha x+x_0=\sqrt{1-\alpha^2y^2}$ is the equation of a circle centered on the $x$ axis.
        \end{enumerate}
    \end{enumerate}
\end{document}