%! Author = Johannes Byle
%! Date = 10/17/2021

% Preamble
\documentclass[12pt]{article}
\title{}
\author{Johannes Byle}

% Preamble
\documentclass[12pt]{article}
\title{Math Methods Assignment \#6}
\author{Johannes Byle}


% Packages
\usepackage{amsmath}
\usepackage[margin=0.75in]{geometry}
\usepackage{lipsum}
\usepackage{bbold}
\newcommand{\p}[2]{\frac{\partial #1}{\partial #2}}
\newcommand{\der}[2]{\frac{d #1}{d #2}}
\newcommand{\curl}{\nabla\times}
\newcommand{\divr}{\nabla\cdot}
% Document
\begin{document}
  \maketitle
  \begin{enumerate}
    \item
    \begin{enumerate}
      \item Starting with Ampere's Law:
      \begin{align*}
        \curl\mathcal{B}&=\frac{4\pi}{c}\pmb{J}+\frac{1}{c}\p{\mathcal{E}}{t}\\
        \divr\left(\nabla\times\mathcal{B}\right)&=\frac{4\pi}{c}\divr\pmb{J}+\frac{1}{c}\p{(\nabla\cdot\mathcal{E})}{t}=0\quad\text{Using the identity: }\nabla\left( \nabla\cdot\times A \right)=0\\
        \divr\left(\nabla\times\mathcal{B}\right)&=\frac{4\pi}{c}\divr\pmb{J}+\frac{4\pi}{c}\p{\rho}{t}=0\quad\text{Using: }\nabla\cdot\mathcal{E}=4\pi\rho\\
        \divr\pmb{J}+\p{\rho}{t}&=0
      \end{align*}
      \item Starting with the curl of the electric field:
      \begin{align*}
        \curl\mathcal{E}&=-\frac{1}{c}\p{\mathcal{B}}{t}\\
        \curl\mathcal{E}&=-\frac{1}{c}\p{(\curl\mathcal{A})}{t}\\
        \curl\left(\mathcal{E}+\frac{1}{c}\p{\mathcal{A}}{t}\right)&=0\\
        \mathcal{E}&=-\nabla\phi-\frac{1}{c}\p{\mathcal{A}}{t}\quad\text{Rewriting in terms of a scalar potential}
      \end{align*}
      \item
      \begin{gather*}
        F=
        \begin{bmatrix}
          0    & -B_3 & B_2  \\
          B_3  & 0    & -B_1 \\
          -B_2 & B_1  & 0
        \end{bmatrix}\\
        B=
        \begin{bmatrix}
          \frac{1}{2}(B_1+B_1) & \frac{1}{2}(B_2+B_2)& \frac{1}{2}(B_3+B_3)
        \end{bmatrix}
      \end{gather*}
      \item Starting with $\mathcal{B}=\curl A$:
      \begin{align*}
        F_{ij}&=\epsilon_{ijk}(\curl A)_k\\
        F_{ij}&=\epsilon_{ijk}\epsilon_{jlm}\partial_l A_m\quad\text{Rewriting the curl using Levi-Civita}\\
        F_{ij}&=(\delta_{il}\delta_{jm}-\delta_{jl}\delta_{im})\partial_l A_m\\
        F_{ij}&=\partial_i A_j-\partial_j A_i
      \end{align*}
      \item Proving the first part:
      \begin{align*}
        F_{4j}&=\p{A_j}{x_4}-\p{A_4}{x_j}\\
        -(\p{A_4}{x_j}-\p{A_j}{x_4})&=\p{A_j}{x_4}-\p{A_4}{x_j}
      \end{align*}
      Assuming this 4th term is time, $\mathcal{E}=-\nabla\phi-\frac{1}{c}\p{\mathcal{A}}{t}$ and $\partial_t A_j=0$ show that $F_{4j}=iE_j$.
      \item Starting with the definition given:
      \begin{align*}
        \partial_i F_{jk}+\partial_j F_{ki}+\partial_k F_{ij}&=0\\
        \partial_i (\partial_j A_j-\partial_k A_j)+\partial_j (\partial_k A_i-\partial_i A_k)+\partial_k (\partial_i A_j-\partial_j A_i)&=0\\
      \end{align*}
    \end{enumerate}
  \end{enumerate}

\end{document}