%! Author = Johannes Byle
%! Date = 9/7/2021

% Preamble
\documentclass[12pt]{article}
\title{Quantum I Assignment \#2}
\author{Johannes Byle}
\newcommand{\op}[1]{\tilde{\mathbf{#1}}}

% Packages
\usepackage{amsmath}
\usepackage[margin=0.75in]{geometry}
\usepackage{lipsum}
\usepackage{braket}

% Document
\begin{document}
    \maketitle
    \begin{enumerate}
        \item[Q.1]
        \begin{enumerate}
            \item
            \begin{gather*}
                \left(\frac{1}{\sqrt{2}}\right)^2+\frac{i}{2}\frac{-i}{2}+\left(\frac{1}{2}\right)^2=1\\
                \ket{\psi_0}\text{ is normalized}\\
                \left(\frac{1}{\sqrt{3}}\right)^2+\frac{i}{\sqrt{3}}\frac{-i}{\sqrt{3}}=\frac{2}{3}\\
                \ket{\psi_1}\text{ is not normalized}\\
            \end{gather*}
            \item
            \begin{gather*}
                \op{P}_0=\ket{\psi_0}\bra{\psi_0}
            \end{gather*}
            Representing this as a matrix:
            \begin{gather*}
                \begin{bmatrix}
                    \frac{1}{2}\ket{u_1}\bra{u_1}         & \frac{-i}{2\sqrt{2}}\ket{u_1}\bra{u_2} & \frac{1}{2\sqrt{2}}\ket{u_1}\bra{u_3} \\
                    \frac{i}{2\sqrt{2}}\ket{u_2}\bra{u_1} & \frac{1}{4}\ket{u_2}\bra{u_2}          & \frac{i}{4}\ket{u_2}\bra{u_3}         \\
                    \frac{1}{2\sqrt{2}}\ket{u_3}\bra{u_1} & \frac{-i}{4}\ket{u_3}\bra{u_2}         & \frac{1}{4}\ket{u_3}\bra{u_3}         \\
                \end{bmatrix}
                \rightarrow
                \begin{bmatrix}
                    \frac{1}{2}         & \frac{-i}{2\sqrt{2}} & \frac{1}{2\sqrt{2}} \\
                    \frac{i}{2\sqrt{2}} & \frac{1}{4}          & \frac{i}{4}         \\
                    \frac{1}{2\sqrt{2}} & \frac{-i}{4}         & \frac{1}{4}         \\
                \end{bmatrix}
            \end{gather*}
            This is hermitian because it is equal to it's complex conjugate transpose.
            \begin{gather*}
                \op{P}_1=\ket{\psi_1}\bra{\psi_1}
            \end{gather*}
            Representing this as a matrix:
            \begin{gather*}
                \begin{bmatrix}
                    \frac{1}{3}\ket{u_1}\bra{u_1} & \frac{-i}{3}\ket{u_1}\bra{u_2} \\
                    \frac{i}{3}\ket{u_2}\bra{u_1} & \frac{1}{3}\ket{u_2}\bra{u_2}
                \end{bmatrix}
                \rightarrow
                \begin{bmatrix}
                    \frac{1}{3} & \frac{-i}{3} \\
                    \frac{i}{3} & \frac{1}{3}
                \end{bmatrix}
            \end{gather*}
            This is hermitian because it is equal to it's complex conjugate transpose.
        \end{enumerate}
        \item[1.9]
        \begin{enumerate}
            \item
            \begin{gather*}
                \prod_{a_i}(A-a')=(A-a_1)(A-a_2)\cdots(A-a_i)
            \end{gather*}
            When this operator operates on some eigenket of A since it's eigenvalue will be in the series, one of the terms will go to zero, making the whole expression zero.
            \item
            \begin{gather*}
                \prod_{a''\neq a'}(A-a')=\frac{(a'-a'')}{(a'-a'')}\frac{(a'''-a'')}{(a'-a'')}\cdots\frac{(a_n-a'')}{(a'-a'')}
            \end{gather*}
            \item
            \begin{gather*}
                S_z=\frac{\hbar}{2}\left[ (\ket{+}\bra{+})-(\ket{-}\bra{-}) \right]\\
                \prod_{a_i}S_z\ket{+}=\left(\frac{\hbar}{2}-\frac{\hbar}{2}\right)\left(\frac{\hbar}{2}-0\right)=0
            \end{gather*}
        \end{enumerate}
        \item[1.11]
        \begin{gather*}
            \op{S}=\left(S_x,\ S_y,\ S_z\right)\\
            \op{n}=\left(\cos\alpha\sin\beta,\ \sin\alpha\sin\beta,\ \cos\beta\right)\\
            \op{S}\cdot\op{n}=\left(S_x\cos\alpha\sin\beta,\ S_y\sin\alpha\sin\beta,\ S_z\cos\beta\right)\\
            \op{S}\cdot\op{n}=
            \frac{\hbar}{2}
            \begin{bmatrix}
                0 & 1 \\
                1 & 0
            \end{bmatrix}
            \cos\alpha\sin\beta+
            \frac{\hbar}{2i}
            \begin{bmatrix}
                0  & 1 \\
                -1 & 0
            \end{bmatrix}
            \sin\alpha\sin\beta+
            \frac{\hbar}{2}
            \begin{bmatrix}
                1 & 0  \\
                0 & -1
            \end{bmatrix}
            \cos\beta\\
            \op{S}\cdot\op{n}=
            \frac{\hbar}{2}\left(
            \begin{bmatrix}
                0                   & \cos\alpha\sin\beta \\
                \cos\alpha\sin\beta & 0
            \end{bmatrix}+
            \begin{bmatrix}
                0                     & i\sin\alpha\sin\beta \\
                -i\sin\alpha\cos\beta & 0
            \end{bmatrix}+
            \begin{bmatrix}
                \cos\beta & 0          \\
                0         & -\cos\beta
            \end{bmatrix}
            \right)\\
            \op{S}\cdot\op{n}=
            \begin{bmatrix}
                \cos\beta                                & \cos\alpha\sin\beta      +i\sin\alpha\sin\beta \\
                \cos\alpha\sin\beta-i\sin\alpha\sin\beta & -\cos\beta
            \end{bmatrix}
        \end{gather*}
        Solving for $\ket{+}$ as $\begin{bmatrix}
                                      a \\
                                      b
        \end{bmatrix}$:
        \begin{gather*}
            \ket{\op{S}\cdot\op{n};+}=
            \begin{bmatrix}
                \cos\beta                                & \cos\alpha\sin\beta      +i\sin\alpha\sin\beta \\
                \cos\alpha\sin\beta-i\sin\alpha\sin\beta & -\cos\beta
            \end{bmatrix}
            \begin{bmatrix}
                a \\
                b
            \end{bmatrix}
            =
            \frac{\hbar}{2}
            \begin{bmatrix}
                a \\
                b
            \end{bmatrix}\\
            \begin{bmatrix}
                \cos\beta             & e^{i\alpha}\sin\beta \\
                e^{-i\alpha}\sin\beta & -\cos\beta
            \end{bmatrix}
            \begin{bmatrix}
                a \\
                b
            \end{bmatrix}
            =
            \frac{\hbar}{2}
            \begin{bmatrix}
                a \\
                b
            \end{bmatrix}\\
        \end{gather*}
        As a system of equations:
        \begin{gather*}
            a\cos\beta+be^{i\alpha}\sin\beta=\frac{\hbar}{2}a\\
            ae^{-i\alpha}\sin\beta  -b\cos\beta=\frac{\hbar}{2}b
        \end{gather*}
        Solving for $a$:
        \begin{gather*}
            b=\frac{a\sin\beta e^{-i\alpha}}{\left(\cos\beta+\frac{\hbar}{2}\right)}\\
            a\left(\cos\beta-\frac{\hbar}{2}\right)+\frac{a\sin\beta e^{-i\alpha}}{\left(\cos\beta+\frac{\hbar}{2}\right)}\left(\sin\beta e^{i\alpha}\right)
        \end{gather*}
        \item[1.17]
        We can show that for any base ket $\ket{\psi_n}$, $AB\ket{\psi_n}=BA\ket{\psi_n}$:
        \begin{gather*}
            AB\ket{\psi_n}=Ab_n\ket{\psi_n}=a_n b_n\ket{\psi_n}=BA\ket{\psi_n}
        \end{gather*}
        Since this is true for any $\ket{\psi_n}$ and we know that the simultaneous eigenkets form a complete orthonormal set of base kets $[A, B]=0$.\footnote{https://en.wikipedia.org/wiki/Complete\_set\_of\_commuting\_observables}
        \item[1.18]
        If $\ket{\psi}$ is a eigenket of both $A$ and $B$ then:
        \begin{gather*}
            A\ket{\psi}=a\ket{\psi}\\
            B\ket{\psi}=b\ket{\psi}\\
            (AB+BA)\ket{\psi}=2ab\ket{\psi}
        \end{gather*}
        The only way this is true is if either $a$ or $b$ is 0.\footnote{http://peeterjoot.com/2015/09/28/can-anticommuting-operators-have-a-simulaneous-eigenket/}
    \end{enumerate}

\end{document}