%! Author = johannes
%! Date = 9/20/21

% Preamble
\documentclass[12pt]{article}
\title{Quantum I Assignment \#5}
\author{Johannes Byle}
\newcommand{\op}[1]{\tilde{\mathbf{#1}}}

% Packages
\usepackage{amsmath}
\usepackage[margin=0.75in]{geometry}
\usepackage{lipsum}
\usepackage{braket}
\usepackage{mathrsfs}
\newcommand{\opx}{\pmb{\text{x}}}
\newcommand{\jac}{\mathscr{J}(d\opx')}
\newcommand{\jacdag}{\mathscr{J}^{\dag}(d\opx')}
\newcommand{\opp}{\pmb{\text{p}}}
\newcommand{\boost}{\mathscr{B}(d\opp')}
\newcommand{\boostdag}{\mathscr{B}^{\dag}(d\opp')}
\newcommand{\opw}{\pmb{\text{W}}}
% Document
\begin{document}
  \maketitle
  \begin{enumerate}
    \item[1.33] Starting with the translation operator applied to the expectation value for $\opx$:
    \begin{gather*}
      \bra{\alpha}\jacdag\opx\jac\ket{\alpha}\\
    \end{gather*}
    By equation 1.207 we know:
    \begin{gather*}
      \opx\jac-\jac\opx=d\opx'
    \end{gather*}
    Since the translation operator is unitary we can apply $\jacdag$ to both sides:
    \begin{gather*}
      \jacdag\left[\opx\jac-\jac\opx\right]=\jacdag d\opx'\\
      \jacdag \opx\jac-\jacdag\jac\opx=\jacdag d\opx'\\
      \jacdag \opx\jac-\opx=\jacdag d\opx'\\
      \opx+d\opx'-\opx=\jacdag d\opx'
    \end{gather*}
    This means that $\bra{\alpha}\jacdag\opx\jac\ket{\alpha}\rightarrow\bra{\alpha}\opx\ket{\alpha}+d\opx'$.\\
    Using the same process for $\opp$:
    By equation 1.227 we know:
    \begin{gather*}
      \opp\jac-\jac\opp=0
    \end{gather*}
    Since the translation operator is unitary we can apply $\jacdag$ to both sides:
    \begin{gather*}
      \jacdag\left[\opp\jac-\jac\opp\right]=0\\
      \jacdag \opp\jac-\jacdag\jac\opp=0\\
      \jacdag \opp\jac-\opp=0\\
      \opp+d\opp'-\opp=0
    \end{gather*}
    This means that $\bra{\alpha}\jacdag\opp\jac\ket{\alpha}\rightarrow\bra{\alpha}\opp\ket{\alpha}$.\\
    \item[1.34] Satisfies unitary property because $\opw$ is hermitian:
    \begin{align*}
      \boostdag\boost&=(1-i\opw\cdot d\opp)(1+i\opw\cdot d\opp)\\
      &=(1-i\opw\cdot d\opp^{\dagger})(1+i\opw\cdot d\opp)\\
      &=1-i(\opw-\opw^{\dagger})\\
      &\simeq 1
    \end{align*}
    Satisfies the associative property:
    \begin{align*}
      \boostdag\mathscr{B}(d\opp'')&=(1+i\opw\cdot d\opp')\cdot(1+i\opw\cdot d\opp'')\\
      &\simeq 1-i\opw\cdot(d\opp' d\opp'')\\
      &= \mathscr{B}(d\opp'+d\opp'')
    \end{align*}
    Satisfies the inverse property trivially:
    \begin{align*}
      \mathscr{B}(-d\opp')&=\mathscr{B}^{-1}(d\opp')\\
      1+i\opw\cdot d\opp&=-(-1-i\opw\cdot d\opp)
    \end{align*}
    Since $d\opp$ has units of $\frac{\text{kg m}}{\text{s}^2}$
    \item[1.35]
  \end{enumerate}


\end{document}