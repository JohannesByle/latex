%! Author = johannes
%! Date = 9/13/21

% Preamble
\documentclass[12pt]{article}
\title{Quantum I Assignment \#3}
\author{Johannes Byle}
\newcommand{\op}[1]{\tilde{\mathbf{#1}}}

% Packages
\usepackage{amsmath}
\usepackage[margin=0.75in]{geometry}
\usepackage{lipsum}
\usepackage{braket}
\usepackage{hyperref}

% Document
\begin{document}
  \maketitle
  \begin{enumerate}
    \item[1.6]
    \begin{enumerate}
      \item
      \begin{gather*}
        \text{tr}(XY)=\sum_i\bra{\psi_i}XY\ket{\psi_i}\\
        \text{tr}(XY)=\sum_{ij}\bra{\psi_i}X\ket{\psi_j}\bra{\psi_j}Y\ket{\psi_i}\\
        \text{tr}(XY)=\sum_{ij}\bra{\psi_j}Y\ket{\psi_i}\bra{\psi_i}X\ket{\psi_j}=\text{tr}(YX)\\
      \end{gather*}
      \item
      From Sakurai page 15:
      \begin{gather*}
        XY\ket{\alpha}=X(Y\ket{\alpha}){\mathop{\leftrightarrow}^_{\text{DC}}} (\bra{\alpha}Y^{\dagger})X^{\dagger}=\bra{\alpha}Y^{\dagger}X^{\dagger}
      \end{gather*}
      \item
      Using the Taylor series expansion:
      \begin{gather*}
        \exp[if(A)]=\exp[if(a_0\ket{\alpha_0})]+i\exp[if(a_0\ket{\alpha_0})]a_1\ket{\alpha_1}f'(a_0\ket{\alpha_0})\cdots
      \end{gather*}
    \end{enumerate}
    \item[1.15]
    From Sakurai page 30:
    \begin{gather*}
      |\braket{c'|b'}|^2|\braket{b'|a'}|^2
    \end{gather*}
    Replacing with our variables:
    \begin{gather*}
      a'=s_z;+,\ b'=s_n;+,\ c'=s_z;-\\
      P=|\braket{s_z;-|s_n;+}|^2|\braket{s_n;+|s_z;+}|^2
    \end{gather*}
    Replacing with matrices:
    \begin{gather*}
      S_z=\frac{\hbar}{2}
      \begin{bmatrix}
        1 & 0  \\
        0 & -1
      \end{bmatrix},\quad S_n=\frac{\hbar}{2}
      \begin{bmatrix}
        \cos\beta & \sin\beta  \\
        \sin\beta & -\cos\beta
      \end{bmatrix}\\
      \ket{+}=
      \begin{bmatrix}
        1 \\
        0
      \end{bmatrix},\quad \ket{-}=
      \begin{bmatrix}
        0 \\
        1
      \end{bmatrix}\\
      P=|
      \begin{bmatrix}
        0 \\
        1
      \end{bmatrix}^T
      \begin{bmatrix}
        1 & 0  \\
        0 & -1
      \end{bmatrix}^T
      \begin{bmatrix}
        \cos\beta & \sin\beta  \\
        \sin\beta & -\cos\beta
      \end{bmatrix}
      \begin{bmatrix}
        1 \\
        0
      \end{bmatrix}|^2|
      \begin{bmatrix}
        1 \\
        0
      \end{bmatrix}^T
      \begin{bmatrix}
        \cos\beta & \sin\beta  \\
        \sin\beta & -\cos\beta
      \end{bmatrix}^T
      \begin{bmatrix}
        1 & 0  \\
        0 & -1
      \end{bmatrix}
      \begin{bmatrix}
        1 \\
        0
      \end{bmatrix}|^2\\
      P=|
      \begin{bmatrix}
        0 & -1\\
      \end{bmatrix}
      \begin{bmatrix}
        \cos\beta \\
        \sin\beta
      \end{bmatrix}|^2|
      \begin{bmatrix}
        \cos\beta & \sin\beta
      \end{bmatrix}
      \begin{bmatrix}
        1 \\
        0
      \end{bmatrix}|^2\\
      P=|-\sin\beta|^2|\cos\beta|^2\\
      P=\sin^2(\beta)\cos^2(\beta)
    \end{gather*}
    This maximum value is then $\beta=\frac{\pi}{4}$.
    \item[1.17]
    We can show that for any base ket $\ket{\psi_n}$, $AB\ket{\psi_n}=BA\ket{\psi_n}$:
    \begin{gather*}
      AB\ket{\psi_n}=Ab_n\ket{\psi_n}=a_n b_n\ket{\psi_n}=BA\ket{\psi_n}
    \end{gather*}
    Since this is true for any $\ket{\psi_n}$ and we know that the simultaneous eigenkets form a complete orthonormal set of base kets $[A, B]=0$
    \item[1.19]
    Assuming $\ket{n}$ is a nondegenerate energy state.
    Since $[A_1,H]=0$, and $HA_1\ket{n}=A_1 H\ket{n}=E_n A_1\ket{n}$ and since $A_1$ and $H$ commute therefore $A_1\ket{n}=a_1\ket{n}$.
    In the same way $A_2\ket{n}=a_2\ket{n}$ but that means that $A_1 A_2\ket{n}=a_1 a_2\ket{n}$ which cannot be true if $A_1 A_2\neq A_2 A_1$.
    The only way this could be true is if either $a_1$ or $a_2$ is 0.
    \item[1.24]
    Modeling the ice pick as an inverse pendulum we get the following equation of motion:\footnote{https://en.wikipedia.org/wiki/Inverted\_pendulum}
    \begin{gather*}
      \ddot{\theta}=\frac{g}{l}\sin\theta
    \end{gather*}
    Since we only care about minuscule movements around the equilibrium point we can make the small angle approximation:
    \begin{gather*}
      \ddot{\theta}=\frac{g}{l}\theta\\
      \theta(t)=c_1 e^{t\sqrt{g/l}}+c_2 e^{-t\sqrt{g/l}}
    \end{gather*}
    Solving for the starting conditions.
    \begin{gather*}
      x_0=\theta(0)l=(c_1+c_2)l\\
      p_0=m\theta(0)l=ml\sqrt{\frac{g}{l}}(c_1-c_2)
    \end{gather*}
    Apparently we can just ignore $c_2$ because it decays exponentially.
    We can now use $x_0$ and $p_0$ to get the uncertainty relation:
    \begin{gather*}
      x_0 p_0=\frac{\hbar^2}{4}=ml^2\sqrt{\frac{g}{l}}c_1^2\\
      c_1=\sqrt{\frac{\hbar^2}{4ml^2}\sqrt{\frac{l}{g}}}
    \end{gather*}
    If we say that the pick becomes unstable at about half a degree of rotation, or around 0.01 radians, a length of 1 m and a mass of 1 kg, we get the following results.
    \begin{gather*}
      t=\log\left(\frac{\theta}{c_1}\right)\sqrt{\frac{l}{g}}\\
      t=\log\left(\frac{0.01}{\sqrt{\frac{(1.05\cdot10^{-34})^2}{4\cdot 1\cdot 1^2}\sqrt{\frac{1}{1}}}}\right)\sqrt{\frac{1}{1}}\approx74\text{ s}
    \end{gather*}
  \end{enumerate}

\end{document}