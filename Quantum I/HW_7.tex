%! Author = Johannes Byle
%! Date = 10/19/2021

% Preamble
% Preamble
\documentclass[12pt]{article}
\title{Quantum I Assignment \#7}
\author{Johannes Byle}


% Packages
\usepackage{amsmath}
\usepackage[margin=0.75in]{geometry}
\usepackage{lipsum}
\usepackage{physics}
\usepackage{bbold}
\usepackage{braket}
\newcommand{\p}[2]{\frac{\partial #1}{\partial #2}}
\newcommand{\der}[2]{\frac{d #1}{d #2}}
\newcommand{\OP}[1]{\tilde{\pmb{\text{#1}}}}
\newcommand{\Fop}[1]{\mathcal{#1}}

% Document
\begin{document}
    \maketitle
    \begin{enumerate}
        \item
        \begin{enumerate}
            \item
            \begin{gather*}
                \OP{H}\ket{\alpha}=
                \begin{bmatrix}
                    1 & 0 & 0 \\
                    0 & 2 & 0 \\
                    0 & 0 & 2
                \end{bmatrix}
                \begin{bmatrix}
                    \frac{1}{\sqrt{2}}\ket{u_1} \\
                    \frac{1}{2}\ket{u_2}        \\
                    \frac{1}{2}\ket{u_3}
                \end{bmatrix}
                =
                \begin{bmatrix}
                    \frac{1}{\sqrt{2}}\ket{u_1} \\
                    \ket{u_2}                   \\
                    \ket{u_3}
                \end{bmatrix}
            \end{gather*}
            The values that can be found are $\left( \frac{1}{\sqrt{2}},\ 1,\ 1 \right)$.
            The probabilities are $\left( \frac{1}{2},\ \frac{1}{4},\ \frac{1}{4} \right)$.\\
            The mean value is:
            \begin{gather*}
                \braket{\OP{H}}=
                \begin{bmatrix}
                    \frac{1}{\sqrt{2}} & 1 & 1
                \end{bmatrix}
                \begin{bmatrix}
                    \frac{1}{\sqrt{2}} \\
                    \frac{1}{2}        \\
                    \frac{1}{2}
                \end{bmatrix}\frac{1}{3}=\frac{1}{2}
            \end{gather*}
            \begin{gather*}
                \Delta\OP{H}=\sqrt{\braket{\OP{H}^2}+\braket{\OP{H}}^2}\\
                \braket{\OP{H}^2}=\bra{\alpha}\OP{H}^2\ket{\alpha}=
                \begin{bmatrix}
                    \bra{u_1}\frac{1}{\sqrt{2}} & \bra{u_2}\frac{1}{2}        & \bra{u_3}\frac{1}{2}\ket{u_3}
                \end{bmatrix}
                \begin{bmatrix}
                    1 & 0 & 0 \\
                    0 & 2 & 0 \\
                    0 & 0 & 2
                \end{bmatrix}
                \begin{bmatrix}
                    1 & 0 & 0 \\
                    0 & 2 & 0 \\
                    0 & 0 & 2
                \end{bmatrix}
                \begin{bmatrix}
                    \frac{1}{\sqrt{2}}\ket{u_1} \\
                    \frac{1}{2}\ket{u_2}        \\
                    \frac{1}{2}\ket{u_3}
                \end{bmatrix}=2.5\\
                \braket{\OP{H}}=\bra{\alpha}\OP{H}\ket{\alpha}=
                \begin{bmatrix}
                    \bra{u_1}\frac{1}{\sqrt{2}} & \bra{u_2}\frac{1}{2}        & \bra{u_3}\frac{1}{2}\ket{u_3}
                \end{bmatrix}
                \begin{bmatrix}
                    1 & 0 & 0 \\
                    0 & 2 & 0 \\
                    0 & 0 & 2
                \end{bmatrix}
                \begin{bmatrix}
                    \frac{1}{\sqrt{2}}\ket{u_1} \\
                    \frac{1}{2}\ket{u_2}        \\
                    \frac{1}{2}\ket{u_3}
                \end{bmatrix}=1.5\\
                \Delta\OP{H}=\sqrt{1.5^2+2.5}
            \end{gather*}
            \item
            \begin{gather*}
                \OP{A}\ket{\alpha}=
                \begin{bmatrix}
                    1 & 0 & 0 \\
                    0 & 0 & 1 \\
                    0 & 1 & 0
                \end{bmatrix}
                \begin{bmatrix}
                    \frac{1}{\sqrt{2}}\ket{u_1} \\
                    \frac{1}{2}\ket{u_2}        \\
                    \frac{1}{2}\ket{u_3}
                \end{bmatrix}
                =
                \begin{bmatrix}
                    \frac{1}{\sqrt{2}}\ket{u_1} \\
                    \frac{1}{2}\ket{u_2}        \\
                    \frac{1}{2}\ket{u_3}
                \end{bmatrix}
            \end{gather*}
            The values that can be found are $\left( \frac{1}{\sqrt{2}},\ 1/2,\ 1/2 \right)$.
            The probabilities are $\left( \frac{1}{2},\ \frac{1}{4},\ \frac{1}{4} \right)$.
            The state vector is:
            $\begin{bmatrix}
                 \frac{1}{\sqrt{2}}\ket{u_1} & \frac{1}{2}\ket{u_2}        & \frac{1}{2}\ket{u_3}
            \end{bmatrix}$.
            \item The state vector is $\Fop{U}(t)\ket{\alpha}=e^{\frac{-i\frac{1}{\sqrt{2}}t}{\hbar}}\ket{u_1}+e^{\frac{-it}{\hbar}}\ket{u_2}+e^{\frac{-it}{\hbar}}\ket{u_3}$
            \item
            \begin{gather*}
                \braket{\OP{A}}=
                \begin{bmatrix}
                    \frac{1}{\sqrt{2}} & \frac{1}{2} & \frac{1}{2}
                \end{bmatrix}
                \begin{bmatrix}
                    e^{\frac{-i\frac{1}{\sqrt{2}}t}{\hbar}} \\
                    e^{\frac{-it}{\hbar}}                   \\
                    e^{\frac{-it}{\hbar}}
                \end{bmatrix}\frac{1}{3}=e^{\frac{-i\frac{1}{\sqrt{2}}t}{\hbar}}\frac{1}{\sqrt{2}}+e^{\frac{-it}{\hbar}} \frac{1}{2} +e^{\frac{-it}{\hbar}}  \frac{1}{2}\\
                \braket{\OP{B}}=
                \begin{bmatrix}
                    \frac{1}{2} & \frac{1}{\sqrt{2}}& \frac{1}{2}
                \end{bmatrix}
                \begin{bmatrix}
                    e^{\frac{-i\frac{1}{\sqrt{2}}t}{\hbar}} \\
                    e^{\frac{-it}{\hbar}}                   \\
                    e^{\frac{-it}{\hbar}}
                \end{bmatrix}\frac{1}{3}=e^{\frac{-i\frac{1}{\sqrt{2}}t}{\hbar}}\frac{1}{2}+e^{\frac{-it}{\hbar}} \frac{1}{\sqrt{2}} +e^{\frac{-it}{\hbar}}  \frac{1}{2}
            \end{gather*}
            \item
            \begin{gather*}
                \OP{A}\ket{\alpha}=
                \begin{bmatrix}
                    1 & 0 & 0 \\
                    0 & 0 & 1 \\
                    0 & 1 & 0
                \end{bmatrix}
                \begin{bmatrix}
                    \frac{1}{\sqrt{2}}\ket{u_1} \\
                    \frac{1}{2}\ket{u_2}        \\
                    \frac{1}{2}\ket{u_3}
                \end{bmatrix}
                =
                \begin{bmatrix}
                    e^{\frac{-i\OP{H}t}{\hbar}}\frac{1}{\sqrt{2}}\ket{u_1} \\
                    e^{\frac{-i\OP{H}t}{\hbar}}\frac{1}{2}\ket{u_2}        \\
                    e^{\frac{-i\OP{H}t}{\hbar}}\frac{1}{2}\ket{u_3}
                \end{bmatrix}\\
                \OP{B}\ket{\alpha}=
                \begin{bmatrix}
                    & 1 & 0 \\
                    1 & 0 & 0 \\
                    0 & 0 & 1
                \end{bmatrix}
                \begin{bmatrix}
                    \frac{1}{2}\ket{u_2}        \\
                    \frac{1}{\sqrt{2}}\ket{u_1} \\
                    \frac{1}{2}\ket{u_3}
                \end{bmatrix}
                =
                \begin{bmatrix}
                    e^{\frac{-i\OP{H}t}{\hbar}}\frac{1}{2}\ket{u_1}        \\
                    e^{\frac{-i\OP{H}t}{\hbar}}\frac{1}{\sqrt{2}}\ket{u_2} \\
                    e^{\frac{-i\OP{H}t}{\hbar}}\frac{1}{2}\ket{u_3}
                \end{bmatrix}
            \end{gather*}
        \end{enumerate}
        \item[2.1]
        Starting with the expression $\der{\OP{A}}=\frac{1}{i\hbar}\left[\OP{A},\OP{H}\right]$:
        \begin{gather*}
            \der{\OP{S}_z}{t}=\frac{1}{ih}\left[\OP{S}_z,\omega \OP{S}_z\right]=\frac{1}{i\hbar}0=0\\
            \der{\OP{S}_x}{t}=\frac{1}{ih}\left[\OP{S}_x,\omega \OP{S}_z\right]=\frac{\omega}{i\hbar}i\hbar S_y=-\omega S_y\\
            \der{\OP{S}_y}{t}=\frac{1}{ih}\left[\OP{S}_y,\omega \OP{S}_z\right]=\frac{\omega}{i\hbar}i\hbar S_x=\omega S_x\\
        \end{gather*}
        Repeating to get the second derivative:
        \begin{gather*}
            \der{^2\OP{S}_z}{t^2}=\der{}{t}\frac{1}{i\hbar}0=0\\
            \der{^2\OP{S}_x}{t^2}=\der{}{t}\frac{\omega}{i\hbar}i\hbar S_y=-\omega^2 S_y\\
            \der{^2\OP{S}_y}{t^2}=\der{}{t}\frac{\omega}{i\hbar}i\hbar S_x=-\omega^2 S_x\\
        \end{gather*}
        These two equations are basic differential equations, so it is clear that $\OP{S}_x=\alpha e^{-i\omega t}$ and $\OP{S}_y=\beta e^{-i\omega t}$
        \item[2.3]
        \begin{enumerate}
            \item Starting with the definition of $\OP{S}\cdot\vec{\OP{n}}$:
            \begin{gather*}
                \OP{S}\cdot\vec{\OP{n}}=\cos\left( \frac{\beta}{2} \right)\ket{+}+\sin\left( \frac{\beta}{2} \right)\ket{-}\\
                \OP{S}_x=\frac{\hbar}{2}\left( \ket{+}+\ket{-} \right)\\
            \end{gather*}
            Applying the time evolution operator:
            \begin{gather*}
                \Fop{U}(t)\ket{\OP{S}_n;+}=\exp\left( \frac{-i\OP{H}t}{\hbar} \right)\cos\left( \frac{\beta}{2} \right)\ket{+}+\exp\left( \frac{-i\OP{H}t}{\hbar} \right)\sin\left( \frac{\beta}{2} \right)\ket{-}\\
                \Fop{U}(t)\ket{\OP{S}_n;+}=\exp\left( \frac{-i\omega t}{\hbar} \right)\cos\left( \frac{\beta}{2} \right)\ket{+}+\exp\left( \frac{-i\omega t}{\hbar} \right)\sin\left( \frac{\beta}{2} \right)\ket{-}\\
                P(t)=|\braket{\OP{S}_x;+|\OP{S}_n}|^2=\frac{1}{2}\left[ \exp\left( \frac{-i\omega t}{\hbar} \right)\cos\left( \frac{\beta}{2} \right)+\exp\left( \frac{-i\omega t}{\hbar} \right)\sin\left( \frac{\beta}{2} \right) \right]^2\\
                P(t)=\frac{1}{2}\left( 1+2\sin\beta\cos\omega t \right)
            \end{gather*}
            \item The probability of being in $\ket{\OP{S}_n;-}=1-\frac{1}{2}\left( 1+2\sin\beta\cos\omega t \right)$
            \begin{gather*}
                \braket{\OP{S}_x}=\frac{\hbar}{2}\left(\frac{1}{2}\left( 1+2\sin\beta\cos\omega t \right) \right)-\frac{\hbar}{2}\left( 1-\frac{1}{2}\left( 1+2\sin\beta\cos\omega t \right) \right)\\
                \braket{\OP{S}_x}=\frac{\hbar}{4}\sin\beta\cos\omega t
            \end{gather*}
            \item
            \begin{gather*}
                \beta\rightarrow0\quad P(t)\rightarrow \frac{1}{2}\quad \braket{\OP{S}_x}\rightarrow 0\\
                \beta\rightarrow\pi/2\quad P(t)\rightarrow \frac{1}{2}\left( 1+2\cos\omega t \right)\quad \braket{\OP{S}_x}\rightarrow \frac{\hbar}{4}\cos\omega t
            \end{gather*}
        \end{enumerate}
        \item[2.4] Starting with the definitions (equations 2.63a-b):
        \begin{gather*}
            \ket{v_e}=\cos\theta\ket{v_1}-\sin\theta\ket{v_2}\\
            \ket{v_{\mu}}=\sin\theta\ket{v_1}+\cos\theta\ket{v_2}
        \end{gather*}
        Applying the time evolution operator:
        \begin{gather*}
            \Fop{U}(t)\ket{v_e}=\exp\left( \frac{-i\OP{H} t}{\hbar} \right)\cos\theta+\exp\left( \frac{i\OP{H} t}{\hbar} \right)\sin\theta\\
            \Fop{U}(t)\ket{v_e}=\exp\left( \frac{-ipc\left( 1+\frac{m^2 c^2}{2p^2} \right) t}{\hbar} \right)\cos\theta+\exp\left( \frac{-ipc\left( 1+\frac{m^2 c^2}{2p^2} \right) t}{\hbar} \right)\sin\theta\\
            P(t)=|\braket{v_e|v_e}|^2=\left( \exp\left( \frac{-ipc\left( 1+\frac{m^2 c^2}{2p^2} \right) t}{\hbar} \right)\cos\theta+\exp\left( \frac{-ipc\left( 1+\frac{m^2 c^2}{2p^2} \right) t}{\hbar} \right)\sin\theta\right)^2\\
            P(v_e\rightarrow v_e)=1-\sin^2 2\theta\sin^2\left( \Delta m^2 c^4\frac{L}{4E\hbar c} \right)
        \end{gather*}
        \item[2.9]
        \begin{enumerate}
            \item
            \begin{gather*}
                \int_{-a}^{a} A^2(x'-a)^2(x'+a)^{2e^{-ikx'}}(x'-a)^2(x'+a)^{2e^{ikx'}}dx'=\frac{256 a^9 A^2}{315}=1\\
                A=\pm\frac{3 \sqrt{35}}{16 a^{9/2}}
            \end{gather*}
            \item
            \begin{gather*}
                \braket{x}=\int_{-a}^{a} A^2(x'-a)^2(x'+a)^{2e^{-ikx'}}x(x'-a)^2(x'+a)^{2e^{ikx'}}dx'=0\\
                \braket{p}=-i\hbar\int_{-a}^{a} A^2(x'-a)^2(x'+a)^{2e^{-ikx'}}\p{}{x}(x'-a)^2(x'+a)^{2e^{ikx'}}dx'=-\hbar k\\
                \braket{x^2}=\int_{-a}^{a} A^2(x'-a)^2(x'+a)^{2e^{-ikx'}}x^2(x'-a)^2(x'+a)^{2e^{ikx'}}dx'=\frac{a^2}{11}\\
                \braket{p^2}=\int_{-a}^{a} A^2(x'-a)^2(x'+a)^{2e^{-ikx'}}\p{^2}{x^2}(x'-a)^2(x'+a)^{2e^{ikx'}}dx'=\hbar\frac{3}{a^2}+k^2
            \end{gather*}
            \item
        \end{enumerate}
    \end{enumerate}
\end{document}