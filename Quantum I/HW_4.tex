%! Author = johannes
%! Date = 9/20/21

% Preamble
\documentclass[12pt]{article}
\title{Quantum I Assignment \#4}
\author{Johannes Byle}
\newcommand{\op}[1]{\tilde{\mathbf{#1}}}

% Packages
\usepackage{amsmath}
\usepackage[margin=0.75in]{geometry}
\usepackage{lipsum}
\usepackage{braket}

% Document
\begin{document}
    \maketitle
    \begin{enumerate}
        \item[1.13]
        Assuming that the observable is $\ket{E}$ with eigenkets $a\ket{1}+b\ket{2}$:
        \begin{gather}
            H\ket{E}=\left(H_{11}\ket{1}\bra{1}+H_{22}\ket{2}\bra{2}+H_{12}[\ket{1}\bra{2}+\ket{2}\bra{1}]\right)(a\ket{1}+b\ket{2})\\
            H\ket{E}=H_{11}a\ket{1}+H_{22}b\ket{2}+H_{12}b\ket{1}+H_{12}a\ket{2}
        \end{gather}
        Since $H\ket{E}=\lambda\ket{E}=\lambda a\ket{1}+\lambda b\ket{2}$:
        \begin{gather}
            \lambda a=H_{11}a+H_{12}b\\
            \lambda b=H_{22}b+H_{12}a
        \end{gather}
        Using substition to solve for $\lambda$:
        \begin{gather}
            \frac{\lambda(a-H_{11})}{H_{12}}=b\\
            \frac{\lambda(a-H_{11})}{H_{12}}(\lambda-H_{22})=H_{12}a\\
            \lambda \to \frac{a H_{11}-a H_{12}+b H_{12}-b H_{22}}{a-b}
        \end{gather}
        \item[1.18]
        If $\ket{\psi}$ is a eigenket of both $A$ and $B$ then:
        \begin{gather*}
            A\ket{\psi}=a\ket{\psi}\\
            B\ket{\psi}=b\ket{\psi}\\
            (AB+BA)\ket{\psi}=2ab\ket{\psi}
        \end{gather*}
        The only way this is true is if either $a$ or $b$ is 0.\footnote{http://peeterjoot.com/2015/09/28/can-anticommuting-operators-have-a-simulaneous-eigenket/}
        \item[1.25]
        \begin{enumerate}
            \item No:
            \begin{gather}
                \begin{bmatrix}
                    b & 0  & 0   \\
                    0 & 0  & -ib \\
                    0 & ib & 0
                \end{bmatrix}
                \begin{bmatrix}
                    1 \\
                    0 \\
                    0
                \end{bmatrix}=b\\
                \begin{bmatrix}
                    b & 0  & 0   \\
                    0 & 0  & -ib \\
                    0 & ib & 0
                \end{bmatrix}
                \begin{bmatrix}
                    0 \\
                    1 \\
                    0
                \end{bmatrix}=-ib\\
                \begin{bmatrix}
                    b & 0  & 0   \\
                    0 & 0  & -ib \\
                    0 & ib & 0
                \end{bmatrix}
                \begin{bmatrix}
                    0 \\
                    0 \\
                    1
                \end{bmatrix}=ib
            \end{gather}
            \item
            \begin{gather}
                \begin{bmatrix}
                    a & 0  & 0  \\
                    0 & -a & 0  \\
                    0 & 0  & -a
                \end{bmatrix}
                \begin{bmatrix}
                    b & 0  & 0   \\
                    0 & 0  & -ib \\
                    0 & ib & 0
                \end{bmatrix}=
                \begin{bmatrix}
                    ab & 0    & 0   \\
                    0  & 0    & iab \\
                    0  & -iab & 0
                \end{bmatrix}\\
                \begin{bmatrix}
                    b & 0  & 0   \\
                    0 & 0  & -ib \\
                    0 & ib & 0
                \end{bmatrix}
                \begin{bmatrix}
                    a & 0  & 0  \\
                    0 & -a & 0  \\
                    0 & 0  & -a
                \end{bmatrix}=
                \begin{bmatrix}
                    ab & 0    & 0   \\
                    0  & 0    & iab \\
                    0  & -iab & 0
                \end{bmatrix}
            \end{gather}
            \item
            Choosing the eigenkets: $\ket{1}$, $\ket{2}$, $\ket{3}$ we get the following eigenvalues:
            \begin{gather}
                ab\ket{1}-iab\ket{2}+iab\ket{3}
            \end{gather}
        \end{enumerate}
        \item[1.28]
        \begin{gather}
            S_z=\frac{\hbar}{2}
            \begin{bmatrix}
                1 & 0  \\
                0 & -1
            \end{bmatrix}\quad
            S_x=\frac{\hbar}{2}
            \begin{bmatrix}
                0 & 1 \\
                1 & 0
            \end{bmatrix}\\
            U=S_x\cdot S_z^{-1}=
            \begin{bmatrix}
                0 & -1 \\
                1 & 0
            \end{bmatrix}
        \end{gather}
        This is consistent with:
        \begin{gather}
            U=+\frac{1}{\sqrt{2}}\ket{+}-\frac{1}{2}\ket{-}
        \end{gather}
    \end{enumerate}

\end{document}